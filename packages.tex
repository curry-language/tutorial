\chapter{Managing Curry Packages}

There are many libraries and tools for Curry
that are not distributed with an available Curry system
but can be accessed via the Curry package system.
Currently, more than one hundred packages are
available.
Some of them contain small libraries, others
contain quite complex tools, like the Curry preprocessor.
The Curry Package Manager (\cpm),\index{CPM}
which is part of the Curry systems \pakcs and \kics,
eases the access and use of these packages.
In the following, the basic use of \cpm is explained.
A detailed description of all features of \cpm
can be found in its
\href{http://curry-lang.org/tools/cpm}{user manual}.

The executable of the Curry Package Manager is named \code{cypm}\pindex{cypm}
and located in the \code{bin} directory of \pakcs or \kics.
When you use \cpm for the first time, execute
%
\begin{curry}
> cypm update
\end{curry}
%
to download a local copy of the index of all available packages.
A list of all packages in this index can be shown by
the \code{list} command:
%
\begin{curry}
> cypm list
Name             Synopsis                                         Version   
----             --------                                         -------   
abstract-curry   Libraries to deal with AbstractCurry programs     3.0.0     
abstract-haskell Libraries to represent Haskell programs in Curry  3.0.0     
addtypes         A tool to add missing type signatures in a Curry  3.0.0     
                 program                                           
...
\end{curry}
%
The command
%
\begin{curry}
> cypm info PACKAGE
\end{curry}
%
can be used to show more information about the package with name
\code{PACKAGE}.

\section{Importing Existing Packages}
\label{sec:importing-packages}

To implement a larger application with Curry,
one probably needs many libraries
from other packages.
In this case, one should write the application as a Curry package
so that all required dependencies can be specified
so that \cpm can manage them.
This is explained in more detail in Section~\ref{sec:writing-packages}.
If one wants to execute a smaller Curry program
which uses a library from some Curry package,
then one can use \cpm's \code{add} command
to add this package as a global dependency and
download and install it globally (i.e., in your user-home environment).
For instance, the package \code{pflp} supports
probabilistic functional logic programming
\cite{ChristiansenDylusTeegen18}.
It can be installed as a global dependency by the command
%
\begin{curry}
> cypm add pflp
\end{curry}
%
The command
%
\begin{curry}
> cypm info pflp --all
\end{curry}
%
shows all information about this package.
In particular, one can see that the module \code{PFLP}
is exported by this package.
Thus, one can use this module in a program
using probabilistic functional logic programming by adding the line
%
\begin{curry}
import PFLP
\end{curry}
%
in the header of the program.
When this program is loaded with \pakcs,
the load path is set in such a way so that
the module \code{PFLP} from package \code{pflp}
can be imported.

Note that this method to use packages should be used only
for small programs or to try some package,
since a specific appropriate version of the package is globally installed.
Actually, each version of each Curry system
has its own global package cache.\footnote{%
Use the command \code{cypm config} and look at \code{HOME_PACKAGE_PATH}
to see the directory of this global cache.
If it is no longer needed or should be initialized,
one can simple delete this directory.}
For larger applications, one should install the required packages
locally (which is described in Section~\ref{sec:writing-packages} below).


\section{Installing Tools}
\label{sec:installing-tools}

Many tools for Curry are available as packages.
Actually, more than thirty tools are available as packages,
e.g., to analyze, test, or verify Curry programs,
generate documentation, web frameworks, etc.
Such packages provide an ``executable'' (its name is shown
by \cpm's \code{info} command).
In order to install a tool provided by \code{PACKAGE},
one can use the command
%
\begin{curry}
> cypm install PACKAGE
\end{curry}
%
This command downloads the package into some internal directory
(\code{\$HOME/.cpm/apps_\ldots}), compiles the tool,
and installs the binary of the tool provided by the package
in \code{\$HOME/.cpm/bin}.
Hence it is recommended to add this directory to your path.

For instance, the most recent version of \cpm
can be installed by the commands
%
\begin{curry}
> cypm update
...
> cypm install cpm
... Package 'cpm-xxx' checked out ...
...
INFO  Installing executable 'cypm' into '/home/joe/.cpm/bin'
\end{curry}
%
Now, the binary \code{cypm} of the most recent \cpm version can be used
if \code{\$HOME/.cpm/bin} is in your path
(before the \code{bin} directory of the Curry system!).


\section{Developing Applications and Packages}
\label{sec:writing-packages}

If one wants to implement an application in Curry which uses
other Curry packages, one should create a \define{package} for
this application in order to specify all dependencies
so that they can be managed by \cpm.
The easiest way to create a new package is to use
\cpm's \code{new} command with the name of the new package:
%
\begin{curry}
> cypm new myproject
\end{curry}
%
This creates a new directory \code{myproject} in which
the application can be developed.
The generated file structure is as follows:
%
\begin{verbatim}
   myproject
   |-- LICENSE
   |-- package.json
   |-- README.md
   |-- src
       |-- Main.curry
\end{verbatim}
%
This represents the minimal structure of a reasonable package
(so that it could later be published).
The file \code{README.md} should contain a description
of the package in plain text or markdown syntax.
The LICENSE file is BSD-3 but could be changed.
The source code of the package must be stored
in directory \code{src} and might of consist of an arbitrary
number of modules.

The file \code{package.json} is the most important one
for managing packages. It is a \href{http://json.org}{JSON} file
containing the metadata of the package,
like its name, version, author, synopsis, etc.
The detailed description of all possible metadata fields
can be found in
\href{http://curry-lang.org/tools/cpm}{CPM's user manual}.
One has to modify this file according to the requirements
of the new application.
In particular, one has to specify dependencies on other packages
by modifying the field \code{dependencies} in the
\code{package.json} file,\footnote{One could also use the command
\ccode{cypm add -d PACKAGE} to add new dependencies in this file.}
e.g.:
%
\begin{verbatim}
   {
     ...,
     "dependencies": {
       "base": ">= 1.0.0, < 2.0.0",
       "json": "~> 1.1.0"
     }
   }
\end{verbatim}
%
After modifying the metadata, one can run
%
\begin{curry}
> cypm install
\end{curry}
%
(inside the package directory \code{myproject})
to resolve, download, and install all dependencies of the current package.
Then one can invoke \pakcs and use the modules of all specified
packages.

When the development of the application is completed,
one can specify the name of the main module and the executable
to be generated in the file \code{package.json}
as field \code{executable}:
%
\begin{verbatim}
   {
     ...,
     "executable": {
       "name": "myexec",
       "main": "Start"
      }
   }
\end{verbatim}
%
The name of the executable must be defined (with key \code{name})
whereas the name of the main module (key \code{main}) is optional
(if it is missing, the name \code{Main} is taken).
The main module must export a function
%
\begin{curry}
main :: IO ()
\end{curry}
%
which starts the application.
If a package contains an \code{executable} specification,
the command
%
\begin{curry}
> cypm install
\end{curry}
%
also compiles the main module and installs the executable
in the default directory
\code{\$HOME/.cpm/bin} (see Section~\ref{sec:installing-tools}).



%%% Local Variables:
%%% mode: latex
%%% TeX-master: "main_pdf"
%%% End:
