\chapter{Getting Started with Curry}

There are different implementations of \curry{}
available\footnote{Check the web page \url{\curryurl} for details.}.
As such we can not describe the use of a Curry system
in general. Some implementations are batch-oriented.  In this case a 
Curry program is compiled into machine code and then executed. 
In this introduction we prefer an implementation that supports an interactive environment which provides faster program development by loading and testing programs within the integrated environment.

\pakcs\index{PAKCS} (Portland Aachen Kiel Curry System)
\cite{Hanus16PAKCS}\footnote{\url{http://www.informatik.uni-kiel.de/~pakcs}}
contains such an interactive environment so that we show the use
of this system here in order to get started with Curry.
When you start the interactive environment of PAKCS
(e.g., by typing \ccode{pakcs}\pindex{pakcs}
as a shell command),
you see something like the following output after the system's
initialization:
\begin{verbatim}
     ______      __       _    _    ______   _______     
    |  __  |    /  \     | |  / /  |  ____| |  _____|   Portland Aachen Kiel
    | |  | |   / /\ \    | |_/ /   | |      | |_____    Curry System
    | |__| |  / /__\ \   |  _  |   | |      |_____  |   
    |  ____| / ______ \  | | \ \   | |____   _____| |   Version 1.14.1 (6)
    |_|     /_/      \_\ |_|  \_\  |______| |_______|   
   
   Curry2Prolog(swi 7.2) Compiler Environment (Version of 2017-03-17)
   (RWTH Aachen, CAU Kiel, Portland State University)
   
   Type ":h" for help (contact: pakcs@curry-language.org)
   Prelude> 
\end{verbatim}
Now the system is ready and waits for some input.
By typing \ccode{:q} (quit) you can always leave the system,
but this is not what we intend to do now.
The prefix of the current input line always shows the currently
loaded module or program. In this case the module
\code{Prelude}\pindex{Prelude}
is loaded during system startup.
The standard system module \code{Prelude} contains
the definition of several predefined functions and data structures
which are available in all Curry programs.
For instance, the standard arithmetic functions like \code{+}, \code{*} etc
are predefined so that we can use the system as a simple calculator
(the input typed by the user is underlined):
\begin{prog}
Prelude> \userinput{3+5*4}
23
\end{prog}
In this simple example you can already see the basic functionality
of the environment: if you type an expression, the system
evaluates this expression to a \emph{value}\index{value}
(i.e., an expression without evaluable functions)
and prints this value as the result. % (with a question mark, see below).
%Now hit the ``enter'' key and you are back to input line mode where
Now you can type additional expressions to be evaluated. For instance,
you can compare the values of two expressions with the
usual comparison operators \code{>}, \code{<}, \code{<=}, \code{>=}:
\begin{prog}
Prelude> \userinput{3+5*4 >= 3*(4+2)}
True
\end{prog}
\code{==} and \code{/=} are the operators for equality and disequality
and can be used on numbers as well as on other datatypes:
\begin{prog}
Prelude> \userinput{4+3 == 8}
False
\end{prog}
% You may wonder why the system always puts a question mark after
% the result and then waits for further input.
% The reason is that Curry cannot only perform functional, thus,
% purely deterministic computations, which yields at most one result;
% Curry subsumes logic languages which are able
% to search for different results. Therefore, arbitrary
% expressions might deliver several solutions and PAKCS
% computes these solutions one after the other. Thus, it prints
% the first solution followed by a question mark and,
% if you type \ccode{;} (followed by
% the ``enter'' key), it computes and prints the next solution and so on.
% We will see later examples for this feature but ignore it
% for the moment. So, just hit the ``enter'' key after a result
% in order to ignore the computation of further solutions.
% 
One may want to use Curry as more than a mere desk calculator. Therefore, we will discuss how to write programs in Curry.
In general, a Curry \emph{program}\index{program}
is a set of function definitions.
The simplest sort of functions are those that do not depend
on any input value, i.e., constant functions.
For instance, a definition like
\begin{prog}
nine = 3*3
\end{prog}
(such definitions are also called \emph{rules}\index{rule}
or \emph{defining equations}\index{defining equation}\index{equation!defining})
defines the name \code{nine} as equal to the value of \code{3*3},
i.e., \code{9}. This means that each occurrence of the name \code{nine}
in an expression is replaced by the value \code{9}, i.e., the value
of the expression \ccode{4*nine} is \code{36}.

Of course, it is more interesting to define functions depending
on some input arguments. For instance, a function to compute the
square value of a given number can be defined by
\begin{prog}
square x = x*x
\end{prog}
Now it is time to make some remarks about the syntax
of Curry (which is actually very similar to
Haskell \cite{PeytonJones03Haskell}).
The names of functions and parameters usually start with
a lowercase letter followed by letters, digits and underscores.
The application of a function $f$ to an expression $e$ is denoted
by juxtaposition, i.e., by ``$f~e$''. An exception are the
\emph{infix operators}%
\index{operator}\index{operator!infix}\index{infix operator}
like \code{+} or \code{*} that can be written between
their arguments to enable the standard mathematical notation
for arithmetic expressions. Furthermore, these operators
are defined with the usual associativity and precedence rules
so that an expression like \ccode{2+3+4*5} is interpreted
as \code{((2+3)+(4*5))}. However, one can also enclose
expressions in parenthesis to enforce the intended grouping.

If we write the definitions of \code{nine} and \code{square}
with a standard text editor
into a file (note that each definition must be written on a separate
line starting in the first column) named \ccode{firstprog.curry}
(\proghref{firstprog}{Program}),
we can load (and compile) the program into our environment
by the command
\begin{prog}
Prelude> \userinput{:l firstprog}
\end{prog}
which reads and compiles the file \ccode{firstprog.curry} and
makes all definitions in this program visible in the environment.
After the successful processing of this program, the environment
shows the prefix to the input line as
\begin{prog}
firstprog>
\end{prog}
indicating that the program \ccode{firstprog} is currently loaded.
Now we can use the definitions in this program in the expressions
to be evaluated:
\begin{prog}
firstprog> \userinput{square nine}
81
\end{prog}
If we change our currently loaded program, we can easily reload
the new version by typing \ccode{:r}. For instance,
if we add the definition \ccode{two = 2} to our file
\ccode{firstprog.curry}, we can reload the program as follows:
\begin{prog}
firstprog> \userinput{:r}
\ldots
firstprog> \userinput{square (square two)}
16
\end{prog}
Functions containing only a single arithmetic expression in the
right-hand side of their defining equations might be useful abstractions
of complex expressions but are generally only of limited use.
More interesting functions can be written using conditional
expressions. A \emph{conditional expression}%
\index{conditional expression}\index{expression!conditional}
has the general form \ccode{if $c$ then $e_1$ else $e_2$}
where $c$ is a Boolean expression (yielding the value \code{True}
or \code{False}). A conditional expression is evaluated by
evaluating the condition $c$ first. If its value is \code{True},
the value of the conditional is the value of $e_1$, otherwise
it is the value of $e_2$. For instance, the following rule
defines a function to compute the absolute value of a number:
\begin{prog}
abs x = if x>=0 then x else -x
\end{prog}
Using recursive definitions, i.e., rules where the defined
function occurs in a recursive call in the right-hand side,
we can define functions whose evaluation requires a non-constant
number of evaluation steps. For instance, the following rule
defines the factorial of a natural number
\proghref{absfac}{Program}:
\begin{prog}
fac n = if n==0 then 1
                else n * fac(n-1)
\end{prog}
Note that function definitions can be put in several lines
provided that the subsequent lines start in a column greater
than the column where the left-hand side starts
(this is also called the \emph{layout}\index{layout rule}
or \emph{off-side}\index{off-side rule} rule for separating
definitions).

You might have noticed that functions are defined by rules
like in mathematics without providing any type declarations.
This does not mean that Curry is an untyped language.  On the contrary, Curry is a \emph{strongly typed language}\index{types}
which means that each entity in a program (e.g., functions, parameters)
has a type and ill-typed combinations are detected
by the compiler. For instance, expressions like
\ccode{3*True} or \ccode{fac False} are rejected by the compiler.
Although type annotations need not be written by the programmer,
they are automatically inferred by the compiler
using a \emph{type inference algorithm}\index{type inference}.
Nevertheless, it is a good idea to write down the types
of functions in order to provide at least a minimal documentation
of the intended use of functions. For instance,
the function \code{fac} maps integers into integers and so
its type can be specified by
\begin{prog}
fac :: Int -> Int
\end{prog}
(\code{Int} denotes the predefined type of integers; similarly
\code{Bool} denotes the type of Boolean values).
If one is interested in the type of a function or expression
inferred by the type inference algorithm, one can show it
using the command \ccode{:t} in PAKCS:
\begin{prog}
absfac> \userinput{:t fac}
fac :: Int -> Int
absfac> \userinput{:t  abs 3}
(abs 3) :: Int
\end{prog}
%
A useful feature of Curry (as well as most functional and logic
programming languages) is the ability to define
functions in a \emph{pattern-oriented style}.\index{pattern}
This means that we can put values like \code{True} or \code{False}
in arguments of the left-hand side of a rule
and define a function by using several rules.  The rule that matches the pattern of left-hand side will be called.   
For instance, instead of defining the negation on Boolean values by
the single rule
\begin{prog}
not x = if x==True then False
                   else True
\end{prog}
we can define it by using two rules, each with a different pattern (here we
also add the type declaration):
\begin{prog}
not :: Bool -> Bool
not False = True
not True  = False
\end{prog}
The pattern-oriented notation becomes very useful in combination
with more complex data structures, as we will see later.

One of the distinguishing features of Curry in comparison
to functional languages is its ability to \emph{search for solutions},
i.e., to compute values for the arguments of functions so that
the functions can be evaluated.
For instance, consider the following definitions of some functions
on Boolean values contained in the prelude (note that
Curry also allows functions defined as infix operators, i.e.,
\ccode{x \&\& y} denotes the application of function \code{\&\&}
to the arguments \code{x} and \code{y}):
\begin{prog}
False \&\& _  =  False
True  \&\& x  =  x
~
False || x  =  x
True  || _  =  True
~
not False   =  True
not True    =  False
\end{prog}
The underscore
\ccode{_}\pindex{_}\index{variable!anonymous}\index{anonymous variable}
occurring in the rules for
\code{\&\&} and \code{||} denotes an arbitrary value,
i.e., such an \emph{anonymous variable} is used for argument
variables that occur only once in a rule.

We can use these definitions to compute the value of a Boolean expression:
\begin{prog}
Prelude> \userinput{True \&\& (True || (not True))}
True
\end{prog}
However, we can do more and use the same functions
to compute Boolean values for some (initially unknown) arguments:
\begin{prog}
Prelude> \userinput{x \&\& (y || (not x))  where x,y free}
\{x=True, y=True\} True
\{x=True, y=False\} False
\{x=False, y=y\} False
\end{prog}
Note that the initial expression contains the
\emph{free variables}\index{free variable}\index{variable!free}
\code{x} and \code{y} as arguments.
To support the detection of typos,
free variables in initial expressions must be explicitly declared
by a \ccode{where\ldots{}free} clause at the end of the expression.
This requirement can be relaxed in PAKCS by turning the
\emph{free variable mode} on by the command \ccode{:set +free}.
In this mode, all identifiers in the initial expression that
are not defined in the currently loaded program are considered as
free variables:
\begin{prog}
Prelude> \userinput{:set +free}
Prelude> \userinput{x \&\& (y || (not x))}
\{x=True, y=True\} True
\{x=True, y=False\} False
\{x=False, y=y\} False
\end{prog}
Free variables denote ``unknown'' values.
They are instantiated (i.e., replaced by some concrete values) so that
the instantiated expression is evaluable. As we have seen above,
replacing both \code{x} and \code{y} by \code{True}
makes the expression reducible to \code{True}. Therefore,
the Curry system shows in the first result line \code{True} together with the
bindings (i.e., instantiations) of the free variables
it has done to compute this value.

In general, there is more than one possibility
to instantiate the arguments, e.g., the Boolean variables
\code{x} and \code{y} can be instantiated to \code{True} or \code{False}.
This leads to different solutions which are printed
one after the other as shown above: there is one instantiation
for \code{x} and \code{y} so that the instantiated expression
evaluates to \code{True}, and there are two instantiations
so that the instantiated expression evaluates to \code{False}.
The last result line shows that the initial expression has the value
\code{False} provided that \code{x} is instantiated to \code{False}
but \code{y} can be arbitrary (i.e., \code{y} is not instantiated).

As we have seen, PAKCS evaluates and shows all the values
(also called solutions if variables are instantiated)
of an initial expression. This default mode can be changed
by the command \ccode{:set +interactive}.
In the \emph{interactive mode}, we are asked after each
computed value how to proceed:
whether we want to see the next value (``yes''), no more values (``no''),
or all values without any further interaction (``all'').
Thus, we can show the values of the initial expression
step-by-step as follows (note that it is sufficient to type
the first letter of the answer followed by the ``enter'' key):
\begin{prog}
Prelude> \userinput{:set +interactive}
Prelude> \userinput{x \&\& (y || (not x))  where x,y free}
\{x=True, y=True\}  True
More values? [Y(es)/n(o)/a(ll)] \userinput{y}
\{x=True, y=False\} False
More values? [Y(es)/n(o)/a(ll)] \userinput{y}
\{x=False, y=y\} False
More values? [Y(es)/n(o)/a(ll)] \userinput{y}
No more values.
\end{prog}
The final line indicates that there are no more values to the
initial expression. This situation can also occur if functions
are partially defined, i.e., there is a call to which no rule
is applicable. For instance, assume that we define the function
\code{pneg} by the single rule
\proghref{bool}{Program}
\begin{prog}
pneg True = False
\end{prog}
then there is no rule to evaluate the call \ccode{pneg False}:
\begin{prog}
bool> \userinput{pneg False}
No more values.
\end{prog}
As we have seen in the Boolean example above,
Curry can evaluate expressions containing free variables
by guessing values for the free variables so that the expression
becomes evaluable (the concrete strategy used by Curry will be
explained later, but don't worry: Curry is based on an optimal
evaluation strategy \cite{AntoyEchahedHanus00JACM} that performs
these instantiations in a goal-oriented manner).
However, we might not be interested to see
all possible evaluations but only those that lead to a required
result. For instance, we might be only interested to compute
instantiations in a Boolean formula so that the formula becomes true,
e.g., as in solving an equation.

For this purpose, Curry offers \emph{constraints}\index{constraint},
i.e., formulas that are intended to be solved (instead of computing
an overall value). One of the basic constraints supported by Curry
is equality, i.e., \ccode{$e_1$ =:= $e_2$} denotes an
\emph{equational constraint}\index{equational constraint}%
\index{constraint!equational}
which is solvable whenever the expressions $e_1$ and $e_2$
(which must be of the same type) can be instantiated so that
they are evaluable to the same value. For instance,
the constraint \ccode{1+4=:=5} is solvable, and
the constraint \ccode{2+3=:=x} is solvable if the variable \code{x}
is instantiated to \code{5}. Now we can compute positive solutions
to a Boolean expression by solving a constraint containing \code{True}
on one side:
\begin{prog}
Prelude> \userinput{(x \&\& (y || (not x))) =:= True  where x,y free}
\{x=True, y=True\} True
\end{prog}
% Note that \ccode{success}\pindex{success}
% denotes the trivial, always satisfiable
% constraint, i.e., a result like \ccode{success} indicates that
% the constraint is satisfied with respect to the computed instantiations.
Curry allows the definition of functions by several rules
and is able to search for several solutions. We can combine
both features to define functions that yield more than one
result for a given input. Such functions are called
\emph{non-deterministic} or \emph{set-valued functions}.%
\index{function!non-deterministic}\index{non-deterministic function}%
\index{function!set-valued}\index{set-valued function}
A simple example for a set-valued function
is the following function \code{choose} which yields
non-deterministically one of its arguments as a result
\proghref{choose}{Program}:
\begin{prog}
choose x y = x
choose x y = y
\end{prog}
With this function we could have several results for a particular call:
\begin{prog}
choose> \userinput{choose 1 3}
1
3
\end{prog}
We can use \code{choose} to define other set-valued functions:
\begin{prog}
one23 = choose 1 (choose 2 3)
\end{prog}
Thus, a call to \code{one23} delivers one of the results \code{1},
\code{2}, or \code{3}. Such a function might be useful
to specify the domain of values for which we want to solve a
constraint. For instance, to search for values $x \in \{1,2,3\}$
satisfying the equation $x+x = x*x$, we can solve
this constraint ($c_1\,\code{\&}\,c_2$ denotes the conjunction
of the two constraints $c_1$ and $c_2$):
\begin{prog}
choose> \userinput{x=:=one23 \& x+x=:=x*x  where x free}
\{x=2\} True
\end{prog}
Set-valued functions are often a reasonable alternative
to flexible functions in order to search for solutions.
The advantages of set-valued functions will become clear
when we have discussed the (demand-driven) evaluation strategy
in more detail.

This chapter is intended to provide a broad overview
of the main features of Curry and the use of an interactive
programming environment so that one can easily try the subsequent
examples. In the next chapter, we will discuss the features
of Curry in more detail.


%%% Local Variables: 
%%% mode: latex
%%% TeX-master: "main"
%%% End: 
