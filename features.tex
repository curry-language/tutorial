\chapter{Main Features of Curry}

\section{Overview}

The major elements declared in program are
\emph{functions} and \emph{data structures}.

\begin{itemize}
\item{} 
A \define{function} or \define{operation} defines
a computation similar to an expression.
However, the expression computed by a function
has a name and is often parameterized.
These characteristics enable you to execute the same computation, possibly
with different parameters, over and over in the same program
by simply invoking the computation's name
and setting the values of its parameters.
A function also provides a \emph{procedural abstraction}.
Rather than coding a computation by means of a possibly
complicated expression, you can factor out portions of this computation
and abstract them by their names.
\item{} 
A \define{data structure} is a way to organize data.
For example, you can record the movements of your bank account
in a column in which deposits are positive numbers and withdrawals
are negative numbers.  Or you can record the same movements in
two columns, one for deposits and another for withdrawals,
in which all numbers are positive.
With the second option,
the columns rather than the signs specialize the meaning of the numbers.
The way in which information is organized may ease some computations,
such as retrieving portions of information, and
is intimately related, through pattern matching,
to the way in which functions are coded.
\end{itemize}
%
This section describes in some detail both of these features and
a number of related concepts.
Curry has some additional features not described in this section.
Since they are useful to support particular programming tasks,
we introduce them later when we discuss such programming techniques.


\section{Expressions}

A function can be regarded as a parameterized expression with a name.
Thus, we begin by explaining what an expression is and
how it is used.
Most expressions are built from simpler subexpressions,
a situation that calls for a recursive, or inductive, definition.
%
\begin{quote}
An \definei{expression}{expression!definition}
is either a symbol or literal value or
is the application of an expression to another expression.
\end{quote}
%
A symbol or literal value is referred to as an \emph{atom}\index{atom}.
For example, numbers and the Boolean symbols \ccode{True}
and \ccode{False} are examples of atoms.
Atoms constitute the most elementary expressions.
These elementary expressions can be combined to create
more complex expressions, e.g., \ccode{2\,+\,3} or
\ccode{not\,\,True}.
The combination is referred to as a
\emph{function application}\index{function!application}.
Since a function application is a very common activity,
it is convenient to denote it as simply as possible.
This convenience is obtained to the extreme by writing the two expressions one
near the other as in \ccode{not\,\,True}.
This notation is referred to as \emph{juxtaposition}\index{juxtaposition}.

In the above expressions, the symbols \ccode{+} and
\ccode{not} are operations.
Both are predefined in the standard library \code{Prelude}\pindex{Prelude}.
Although conceptually the symbols \ccode{+} and \ccode{not}
are alike, syntactically they differ.
The symbol \ccode{+} is a \emph{infix operator}
as in the ordinary mathematical notation.  
Infix operators have a \emph{precedence}\index{infix operator!precedence}
and an \emph{associativity}\index{infix operator!associativity}
so that the expression \ccode{2\,+\,3\,*\,4}
is understood as \ccode{2\,+\,(3\,*\,4)} and
the expression \ccode{4\,-\,3\,-\,2} is understood
as \ccode{(4\,-\,3)\,-\,2}.
The precedence and associativity of an infix symbol
are defined in a program by a declaration.
The following declarations, from the prelude,
define these parameters for some ordinary arithmetic operations:
%
\begin{curry}
infixl 7 *, `div`, `mod`
infixl 6 +, -
infix  4  <, >, <=, >=
\end{curry}
%
For example, 
the precedence of the addition and subtraction operators is 6
and their associativity is left.
The relational operators have precedence 4 and are not associative.
Operators with a higher precedence bind stronger, i.e.,
the expression \ccode{4\,<\,2\,+\,3} is interpreted as
\ccode{4\,<\,(2\,+\,3)}.\index{infix operator!precedence}
%
\begin{quote}
\emph{Infix declarations}\index{infix operator!declaration}
must always occur at the beginning of a program.
The precedence of an operator\index{precedence!values}
is an integer between 0 and 9 inclusive.
The associativity of an operator\index{associativity!values} is either
\emph{left}, denoted by the keyword \ccode{infixl}\pindex{infixl}
or \emph{right}, denoted by the keyword \ccode{infixr}\pindex{infixr}.
Non-associative infix operators are declared using the 
keyword \ccode{infix}\pindex{infix}.
\end{quote}
%
Most often, an infix operator is any user-defined sequence of characters
taken from the set
``\verb`~!@#$%^&*+-=<>?./|\:`''\index{infix operator!character set}.
% $ to balance groups
Alphanumeric identifiers can be defined and used as infix operators
if they are surrounded by backquotes, as \ccode{`div`} and
\ccode{`mod`} in the previous declaration.
For example, for any integer value $x$, the following expression
evaluates to $x$ itself.
%
\begin{prog}
$x$ `div` 2 * 2 + $x$ `mod` 2
\end{prog}
%
Non-infix symbols are \emph{prefix}\index{prefix}.
They are applied by prefixing them to their arguments as in
\ccode{not\,\,True}.

\begin{exercise}
\label{ex_factor}
Define a predicate, read as ``factors'' and denoted by
the infix operator ``\verb|./.|'', that tells whether an
integer is a factor of another integer.
The predicate should work for every input and \code{0}
should not be a factor of any integer.
The operator should be non-associative and have precedence 7.
\proghref{exfactor}{Answer}
\end{exercise}

A symbol, whether infix or prefix, can only be applied to
values of an appropriate type.
As one would expect, the Boolean negation operator can be
applied only to a Boolean value.
For example, the expression \ccode{not\,\,2} is an error.
The compiler/interpreter would
report that the expression is incorrectly typed.
We will discuss types in more detail after presenting
data declarations.

The application of an expression to another is a binary operation.
The expression that is being applied is referred to as the
\emph{function} of the application.
The other expression is referred to as the \emph{argument}.
Thus, in \ccode{not\,\,True}, \ccode{not} is the function
and \ccode{True} is the argument.
The situation is slightly more complicated for infix operations.
The reading of \ccode{2\,+\,3} is that the function \ccode{+}
is applied to the expression \ccode{2}.
The result is a function which is further applied to the expression
\ccode{3}.

Expressions can also be conditional, i.e., depend
on the value of a Boolean expression.
Such \emph{conditional expressions}%
\index{conditional expression}\index{expression!conditional}%
\pindex{if-then-else}
have the form \ccode{if $b$ then $e_1$ else $e_2$}.
The value of this expression is the value of $e_1$ if $b$ evaluates
to \code{True}, or the value of $e_2$ if $b$ evaluates to \code{False}.
Thus, the value of \ccode{if 3>4 then 2*2 else 3*4} is \code{12}.


\section{Predefined Types}
\label{Predefined Types}

A \emph{type}\index{type} is a set of values.
Ubiquitous types, such as integers or characters,
are predefined by most programming languages.
Curry makes no exception.
These types are referred to as \emph{builtin}\index{type!builtin}
and are denoted with a familiar, somewhat special, syntax.
Both the availability of builtin types and their characteristics
may depend on a specific implementation of Curry.
The following table summarizes some types available in \pakcs.
%
\begin{center}
\begin{tabular}{@{} || l | l | l || @{}}
\hline \hline
\bf Type & \bf Declaration & \bf Examples \\
\hline
\hline
Integer & \code{Int} & \verb|...,-2,-1,0,1,2,...| \\
\hline
Boolean & \code{Bool} & \code{False, True} \\
\hline
Character & \code{Char} & \verb|'a','b','c',...,'\n',...| \\
\hline
String & \code{String} & \code{"hello", "world"} \\
\hline
List of $\tau$ & \code{[$\tau$]} & \code{[], [0,1,2], 0:1:2:[]} \\
\hline
Unit & \code{()} & \code{()} \\
\hline \hline
\end{tabular}
\end{center}
\pindex{Int}\pindex{Bool}\pindex{Char}%
\pindex{String}\index{List}\index{unit type}\pindex{()}
%
The details of these types are found in the {\pakcs} User Manual.
Below, we only outline a few crucial characteristics of the builtin types.
The integers have arbitrary precision.
Some frequently used non-printable characters are denoted,
as in other popular programming languages,
by escape sequences, e.g., \emph{newline} is denoted by \verb|\n|.
The type \emph{List}\index{list} represents sequences of values.
This type is polymorphic, i.e.,
for any type $\tau$, the type list of $\tau$,
denoted by \ccode{[$\tau$]}, is a type whose instances
are sequences of instances of $\tau$.
The last two examples in the \emph{List} row of the table
denote a list of integers,
their type denoted by \ccode{[Int]}.
The notation of lists will be further discussed later.
The symbol \ccode{()}\pindex{()} denotes the unit type as well as the only
element of this type. The unit type is useful in situations
where the return value of a function is not important.
Another useful type available in {\pakcs},
the \emph{tuple}\index{tuple}, will be described later.


\section{Predefined Operations}
\label{Predefined Operations}

Many frequently-used functions and infix operators,
similar to frequently-used types, are predefined in Curry.
Some of these can be found in the \ccode{Prelude},
a Curry source program automatically loaded when the
compiler/interpreter starts.
A few others are so fundamental that they are built into the language.
Some of these functions and operators are shown in the following table.
%
\begin{center}
\begin{tabular}{@{} || l | c | c | c | l || @{}}
\hline \hline
\bf Description & \bf Ident. & \bf Fix. & \bf Prec. & \bf Type \\
\hline \hline
Boolean equality & {\tt ==} & & 4 & {\tt a -> a -> Bool} \\
\hline
Constrained equality & {\tt =:=} & & 4 & {\tt a -> a -> Bool} \\
\hline
Boolean conjunction & {\tt \&\&} & R & 3 & {\tt Bool -> Bool -> Bool} \\
\hline
Boolean disjunction & {\tt ||} & R & 2 & {\tt Bool -> Bool -> Bool} \\
\hline
Parallel conjunction & {\tt \&} & R & 0 & {\tt Bool -> Bool -> Bool} \\
\hline
Constrained expression & {\tt \&>} & R & 0 & {\tt Bool -> a -> a} \\
\hline
\hline
\end{tabular}
\end{center}
\pindex{==}\pindex{=:=}\pindex{\&\&}\pindex{\char124\char124}\pindex{\&}\pindex{\&>}

\vspace{3ex}
\noindent
Because of non-determinism and free variables, in the following discussion,
different evaluations of the same expression may produce different values.
\\[1ex]
The \emph{Boolean equality}\index{Boolean equality}\index{equality!Boolean}
applied to expressions $u$ and $v$, i.e.,
$u\;\mbox{\tt ==}\;v$, returns \ccode{True}
when $u$ and $v$ evaluate to the
same \emph{value} and \ccode{False} when they
evaluate to different \emph{values}---a
more precise definition will be given later. 
If the evaluation of $u$ and/or $v$ ends in an expression that
still contains functions, e.g., \code{1 `div` 0}, the computation
\emph{fails} and no value is returned.
\\[1ex]
The \emph{constrained equality}\index{constrained equality}%
\index{equality!constrained}
applied to expressions $u$ and $v$, i.e.,
$u\;\mbox{\tt =:=}\;v$ returns \ccode{True}
when $u$ and $v$ evaluate to the
same \emph{value}---a precise definition will be given later. 
Otherwise, the computation \emph{fails} and no value is returned.
A key difference between the Boolean and the constrained equalities
is how they evaluate expressions containing variables.
This will be discussed in some detail in Section~\ref{Logic variables}.
\\[1ex]
The \emph{Boolean conjunction}\index{Boolean conjunction}%
\index{conjunction!Boolean}
applied to expressions $u$ and $v$, i.e.,
$u\;\mbox{\tt \&\&}\;v$, returns \ccode{True}
when $u$ and $v$ evaluate to \ccode{True}.
\\[1ex]
The \emph{Boolean disjunction}\index{Boolean disjunction}%
\index{disjunction!Boolean}
applied to expressions $u$ and $v$, i.e.,
$u\;\mbox{\tt ||}\;v$, returns \ccode{True}
when $u$ or $v$ evaluate to \ccode{True}.
\\[1ex]
The \emph{parallel conjunction}\index{parallel conjunction}%
\index{conjunction!parallel}
applied to expressions $u$ and $v$, i.e.,
$u\;\mbox{\tt \&}\;v$, evaluates $u$ and $v$ concurrently.
If both succeeds, the evaluation succeeds;
otherwise it fails.
\\[1ex]
The \emph{constrained expression}\index{constrained expression}%
\index{expression!constrained}
applied to a constraint $c$ and an expression $e$, i.e.,
$c\;\mbox{\tt \&>}\;e$, evaluates first $c$ and, if $c$ evaluates
to \ccode{True}, then the result is the value of $e$, otherwise it fails.

Curry predefines many more functions and operations, e.g.,
the standard arithmetic and relational operators on numbers.
A complete list can be found both in the Report and the \ccode{Prelude}.

\section{Functions}

\subsection{Basic Concepts}
\label{Basic concepts}

A program function abstracts a function in the mathematical sense.
A function is a device that takes arguments and returns a result.
The result is obtained by evaluating an expression
which generally involves the function's arguments.
The following function computes the \emph{square} of a number.
%
\begin{curry}
square x = x * x
\end{curry}
%
The symbols \ccode{square} is the name or
\emph{identifier}\index{function!identifier} of the function.
The symbol \ccode{x} is the function's 
\emph{argument}\index{function!argument}.
The above declaration is referred to as a
\emph{rewrite rule}\index{rewrite rule}, or simply a rule,
defining a function.
The portion of the declaration to the left of the symbol \ccode{=}
is the rule's \emph{left-hand side}\index{rewrite rule!left-hand side}.
The expression \ccode{x\,*\,x} is the rule's
\emph{right-hand side}\index{rewrite rule!right-hand side}.

When the \ccode{square} symbol is applied to an expression,
e.g., \ccode{2\,+\,3}, this expression is
\emph{bound}\index{function!argument binding}
to the argument \ccode{x}.
The result of the application is
\ccode{(2\,+\,3)\,*\,(2\,+\,3)}, i.e., the body
in which the argument is replaced by its binding.
Thus:
%
\begin{prog}
Prelude> \userinput{square (2+3)}
25
\end{prog}
%
Functions can be \emph{anonymous}\index{function!anonymous},
i.e., without a name.
An anonymous function is useful when a function is referenced
only once.  In this case, the reference to the function
can be replaced by the expression defining the function.
In the following example:
%
\begin{curry}
result = (\x -> x * x) (2+3)
\end{curry}
%
the value of \code{result} is \code{25}.
It is obtained by applying the expression
\code{(\char92x -> x * x)}, an anonymous function,
to \code{(2+3)}, its argument.
An anonymous function definition has the following structure:
%
\begin{curry}
\ <arguments> -> <right-hand side>
\end{curry}
%
The arguments are listed as in any rewrite rule.
The right-hand side is the expression to be evaluated
when the anonymous function is applied to actual arguments.
A more motivating example of anonymous function
is presented in Section~\ref{Higher-Order}

The evaluation\index{evaluation} of any expression,
in particular of a function application,
is \emph{lazy}\index{evaluation!lazy}.
This means that the computation of any expression,
including the subexpressions of a larger expression,
is delayed until the expression's value is actually needed.
The exact meaning of ``actually needed'' is quite technical,
but the intuitive meaning suffices for our purposes.
Many programming languages, such as C and Java,
adopt this evaluation strategy, under the name of
\emph{short circuit}\index{evaluation!short circuit},
only for Boolean expressions.

We will discuss this issue in more detail later.
Although the lazy evaluation strategy is conceptually simpler
than any other strategy, many traditional programming languages
evaluate the arguments of a function call eagerly, i.e.,
before applying a function to its arguments.
This fact is sometimes a source of confusion for the beginner.

\subsection{Pattern Matching}
\label{Pattern Matching}

The definition of a function can be broken into several
rules\index{rewrite rule}.
A single rule would suffice in many cases.
However, several rules allows a definition style,
called \emph{pattern matching}\index{pattern matching},
which is easier to code and understand.
This feature allows a function to dispatch the expression
to be returned depending on the values of its arguments.
The following example shows the definition of the Boolean 
negation function \ccode{not}:
%
\begin{curry}
not True = False
not False = True
\end{curry}
%
The above definition is equivalent to the following one
which does not use pattern matching but relies on a conditional
expression:
%
\begin{curry}
not x = if x == True then False else True
\end{curry}
%
Pattern matching is particularly convenient for functions
that operate on algebraic datatypes.
We will further discuss this aspect after discussing
data declarations.

\subsection{Conditions}

Each rule defining a function
can include one or more \emph{conditions}\index{rule!conditional}.
For Boolean conditions, a rule has the following general
structure\index{rewrite rule!structure}:
%
\[
\begin{array}{@{} l @{\;|\;} c @{\;=\;} c @{}}
\emph{functId} \;\; arg_1 \ldots arg_m & cond_1 & expr_1 \\
                           & \ldots & \ldots \\
                           & cond_n & expr_n
\end{array}
\]
%
A condition is tested after binding
the arguments of a call to the corresponding arguments
in the left-hand side of the rule.
The function is applied to the arguments only if the condition holds.
%
Each condition $cond_i$ is an expression of type \code{Bool}.
The conditions are tested in their textual order.
Thus, the first right-hand side with a condition evaluable to \code{True}
is taken. Furthermore, the last condition can be
\ccode{otherwise}\pindex{otherwise} which is equivalent to \code{True},
i.e., it holds regardless of any value of the arguments.
%
The following example shows a plausible definition of the maximum
of two numbers:
%
\begin{curry}
max x y | x < y     = y
        | otherwise = x
\end{curry}
%
% A rule can also have a
% \emph{constraint}\index{constraint}\index{expression!constraint}
% (i.e., an expression of type \emph{Success}) as a condition.
% In this case, the constraint is checked for satisfiability
% in order to apply the rule. Thus, the function call reduces
% to the right-hand side only if the constraint is satisfied, otherwise
% it fails. Note that multiple conditions as above are not allowed
% for constraint conditions.


\subsection{Non-determinism}
\label{Non-determinism}

Operations can be
\emph{non-deterministic}\index{operation!non-deterministic}%
\index{non-deterministic operation}.
Non-deterministic operations are not functions in the mathematical sense
because they can return different values for the same input.
For example, a hospital's information system defines which
days a doctor is on-call with a non-deterministic function:
%
\begin{curry}
oncall Joan    = Monday
oncall Joan    = Wednesday
oncall Richard = Monday
oncall Luc     = Tuesday
...
\end{curry}
%
The value of \ccode{oncall Joan} can be either
\ccode{Monday} or \ccode{Wednesday}.
The programmer cannot select which of the two values will be computed.
Non-deterministic operations support a programming style similar
to that of logic programs, while preserving some advantages of functional
programs such as expression nesting and lazy evaluation.
In particular, some strong properties concerning the evaluation
of ordinary functions
hold also for non-deterministic operations~\cite{Antoy97ALP}.
For example, suppose that \ccode{today} holds which day of
the week is today.
A predicate, \ccode{available}, telling whether
its argument, a doctor, is available at the current time is coded as:
%
\begin{curry}
available x | oncall x == today = True
            | otherwise         = False
\end{curry}
%
Without non-determinism, coding \ccode{oncall}
would require some data structure, e.g., 
the list of days in which each doctor is on-call,
and defining \ccode{available} would become more complicated.

Non-determinism is a powerful feature.
In programming,
as in other aspects of life, power must be exercised with some care.
A non-deterministic program is appropriate 
only if all its possible outputs are equally desirable.
If some outputs are more desirable than others,
the program should be (more) deterministic.
In this case,
non-determinism could be conveniently used internally by the program
to generate plausible results
which can then be selected according to desirability.

\begin{exercise}
\label{ex_nondettask}
In a manufacturing plant two specialized tasks,
\code{cut} and \code{polish},
are executed only by specialized workers,
\code{Alex}, \code{Bert} and \code{Chuck}.
Not every worker can execute every task.
Only \code{Alex} and \code{Bert} are able to
\code{Cut}, whereas only \code{Bert} and \code{Chuck}
are able to \code{Polish}.
Code a non-deterministic operation, \code{assign},
that assigns to a task a worker that can execute it.
\proghref{exnondettask}{Answer}
\end{exercise}

\subsection{Functional Patterns}
\label{Functional Patterns}

Pattern matching, see Sect.~\ref{Pattern Matching}, is a feature that
provides in a compact and readable form both case distinction
and (sub)argument selection.
The arguments in a rule defined by pattern matching are expressions
consisting of variables and/or data constructor symbols.
Curry amplifies this feature with functional patterns.
In a \emph{functional pattern}\index{functional pattern}\index{pattern!functional}
some argument in a rule defining
a function is an expression containing a function symbol.
For example, the function computing the last element
of a non-empty list can be defined as:
%
\begin{curry}
last (_++[e]) = e
\end{curry}
%
where \ccode{++} is an infix operator that concatenates two lists,
see function \code{conc} in Sect.~\ref{sec-lists}.
The intuition is that if a list $l$ can be seen as the concatenation
of some uninteresting list and a list containing the single element $e$,
then $e$ is the last element of $l$.

The meaning of the above rule is the same as the infinite set of rules:
%
\begin{curry}
last [e] = e
last [_,e] = e
last [_,_,e] = e
...
\end{curry}

\noindent
Code employing functional patterns should be preferred to
similar code using conditions (see below)
because it is more readable and more efficient:
%
\begin{curry}
last xs | xs =:= _++[e] = e  where e free
\end{curry}
%
\begin{exercise}
Define a function that takes a list $a$ of integers and
computes a sublist $l$ of $a$ such that the last element of $l$
is twice the first element.  E.g.,
given the list $[3,6,2,1,4,5]$ the sublists satisfying the
required constraint are $[3,6]$ and $[2,1,4]$.
\proghref{consublist}{Answer}
\end{exercise}

\subsection{Default Rule}
\label{Default Rule}

The Curry language specifies that the textual order of the rules
defining a function is irrelevant in the sense that every rule that is
applicable to an expression should be applied.
In practice, the situation is more delicate.
For example, in \pakcs{} when multiple rules are applicable
some rule is going to be applied before some other
and if the application of the first rule
does not terminate the second rule is never applied.

When defining a function,
Curry allows the programmer to define a rule, called
a \emph{default} rule\index{rule!default}\index{default rule}
\cite{AntoyHanus17TPLP}
that is applied only if all the other rules,
that in this context we call \emph{standard}, cannot be applied.
The processing of default rules require the Curry preprocessor.
It is not part of \pakcs but the preprocessor can easily be installed
by the Curry package manager with the following commands
(see also Section~\ref{sec:installing-tools}):
%
\begin{curry}
> cypm update
> cypm install currypp
\end{curry}
%
After the installation of the Curry preprocessor,\footnote{%
The executable \code{currypp} of the preprocessor is stored in
\code{\$HOME/.cpm/bin} so that one should have this directory
in the load path.}
one can use default rules in a program by placing the following
line at the beginning of the source program.
%
\begin{curry}
{-# OPTIONS_CYMAKE -F --pgmF=currypp --optF=defaultrules #-}  
\end{curry}
%
We recall that a standard rule $r$ is applicable to an expression
$e$ iff $e$ and the left-hand side of $r$ unify and the condition
of $r$, if any exists, is satisfied by the instantiation of $e$.
For example, the operation \code{zip}
is defined with a default rule as follows:
%
\begin{curry}
zip (x:xs) (y:ys) = (x,y) : zip xs ys
zip'default _ _ = []
\end{curry}
%
Since the standard rule is applicable to \code{zip\,[1]\,[2]}, the
default rule is ignored so that this expression is solely reduced to
\code{(1,2):zip\,[]\,[]}. Since the standard rule is not applicable to
\code{zip\,[]\,[]}, the default rule is applied and yields the value \code{[]}.
Altogether, the only value of \code{zip\,[1]\,[2]} is \code{[(1,2)]}.
However, if
some argument has more than one value, we use the evaluation principle
above for each combination.  Thus, the call \code{zip\,([1]\,?\,[])\,[2]} yields
the two values \code{[(1,2)]} and \code{[]}.

Default rules are useful to regain control after a failed search.  For
example, consider looking up the value of a key in a key-value pair list.
%
\begin{curry}
  lookup (key, _ ++ [(key,value)] ++ _) = Just value
  lookup'default _ = Nothing
\end{curry}
%
If the key of a search is not in a list $l$, the call
\code{lookup} $l$ returns \code{Nothing} instead of failing.

\paragraph{Important notes:}
\begin{enumerate}
\item
Default rules can only be added to operations defined at the top-level
(i.e., not to locally defined operations,
see Section~\ref{sec:localdefinitions}).
A reason for this restriction is that default rules are applied
after searching for all possibilities to apply a previous standard rule.
With local definitions, the precise scope of the ``previous'' search
is difficult to define.

\item
The implementation of default rules is based on set functions
(implemented by the module \code{Control.Search.SetFunctions}).
Therefore, this module should be imported in programs using
default rules.
\end{enumerate}


\section{User-defined Types}

A \emph{type}\index{type} is a set of values.
Some common types, presented in Section~\ref{Predefined Types},
are built into the language
and the programmer does not declare them.
All other types used in a program must be declared by the programmer.
The classification of some types as builtin vs.~user-defined 
is only a matter of convenience.
Builtin and user-defined types are conceptually very similar.
In fact,
the declaration of some builtin types could have been left to the programmer.
For example\index{data declaration}:
%
\begin{curry}
data Boolean = False | True
\end{curry}
%
is exactly how the builtin \ccode{Boolean} type would be declared
if it were not builtin.
In this declaration, the identifier \ccode{Boolean}
is referred to as a \emph{type constructor}%
\index{data declaration!type}\index{type},
whereas the identifiers \ccode{False} and \ccode{True}
are referred to as \emph{data constructors}%
\index{data declaration!data constructor}\index{data constructor}.
The following declarations, very similar to the previous one,
define plausible types \ccode{WeekDay} and \ccode{PrimaryColor}.
%
\begin{curry}
data WeekDay = Monday | Tuesday | Wednesday | Thursday | Friday

data PrimaryColor = Red | Green | Blue
\end{curry}
%
All these types are finite, i.e., they define a finite set of values,
and resemble enumerated types in the Pascal or C languages.

The declaration of an infinite type is similar, but as one should expect,
must be (directly or indirectly) recursive.
The following declaration defines a binary tree of integers.
We recall that the typical definition
of this type says that a \emph{binary tree}\index{binary tree}
is either a \emph{leaf} or it is a \emph{branch} consisting of
two binary trees.
Not surprisingly, this definition is recursive which accounts
for an infinity of trees.
The words ``leaf'' and ``branch'' are conventional names used
to distinguish the two kinds of trees and have no other implicit meaning.
Often, branches include a \emph{decoration}, a value of
some other arbitrary type.
If a tree $T$ is a branch, the two trees in the branch
are referred to as the left and right children of $T$.
A declaration defining binary trees where the decoration
is an integer follows:
%
\begin{curry}
data IntTree = Leaf | Branch Int IntTree IntTree
\end{curry}
%
All the following expressions are values of type \ccode{IntTree}:
%
\begin{curry}
Leaf
Branch 0 Leaf Leaf
Branch 7 (Branch 5 Leaf Leaf) (Branch 9 Leaf Leaf)
\end{curry}
%
The first tree is a leaf and therefore it contains no decoration.
The second tree contains a single decoration, \ccode{0},
and two children both of which are leaves.
The third tree contains three decorations.
Binary trees are interesting because many efficient searching
and sorting algorithms are based on them.

User-defined types can be parameterized by means
of other types similar to the builtin type list
introduced in Section~\ref{Predefined Types}.
These types are called \emph{polymorphic}\index{type!polymorphic}.
For example, if the type of the decoration of a binary tree
is made a parameter of the type of the tree, 
the result is a polymorphic binary tree.
This is achieved by the following declaration
\proghref{BinTree}{Program}:
%
\begin{curry}
data BinTree a = Leaf | Branch a (BinTree a) (BinTree a)
\end{curry}
%
The identifier \ccode{a} is a \emph{type variable}%
\index{data declaration!type variable}\index{type variable}.
Observe that the type variable not only defines the type of the
decoration, but also the type of the subtrees occurring in a branch.
In other words, the type that parameterizes a tree also
parameterizes the children of a tree.
The type variable can be implicitly or explicitly 
bound to some type, e.g., \ccode{Int}
or \ccode{WeekDay} defined earlier.
For example, a function that looks for the string \ccode{Curry}
in a tree of strings is defined as
\proghref{BinTree}{Program}:
%
\begin{curry}
findCurry Leaf           = False
findCurry (Branch x l r) = x == "Curry" || findCurry l || findCurry r
\end{curry}
%
The type of the argument of function \ccode{findCurry} is
\ccode{BinTree String}.
The binding of type \ccode{String} to the type variable
of the definition of the polymorphic type \ccode{BinTree}
is automatically inferred from the definition of function
\ccode{findCurry}.

A polymorphic type such as \ccode{BinTree} can be specialized
by binding its variable to a specific type by an explicit declaration
as follows
\proghref{BinTree}{Program}:
%
\begin{curry}
type IntTree = BinTree Int
\end{curry}
%
where \ccode{type}\pindex{type} is a reserved word of the language.
This declaration defines \ccode{IntTree} as a synonym
of \ccode{BinTree Int}.
The synonym can be used in \emph{type declarations}\index{type!synonym}
to improve readability.
The following example defines a function that tallies
all the decorations of a tree of integers
\proghref{BinTree}{Program}:
%
\begin{curry}
total :: IntTree -> Int
total Leaf           = 0
total (Branch x l r) = x + total l + total r
\end{curry}
%
\begin{exercise}
\label{ex_userlist}
Pretend that list is not a builtin type, with
special syntax, of the language.
Define your own type list.
Define two functions on this type,
one to count how many elements are in a list,
the other to find whether some element is in a list.
\proghref{exuserlist}{Answer}
\end{exercise}

\section{Lists}
\label{sec-lists}

The type list is builtin or predefined by the language.
This type could be easily defined by the programmer,
see Exercise~\ref{ex_userlist},
except that the language allows the representation of lists in a special
notation which is more agile than that that would be
available to the programmer.
The following statement defines important concepts of a list:
%
\begin{quote}
A \emph{list}\index{list!definition} is either \emph{nil}\index{list!nil}
or it is a \emph{cons}\index{list!cons} consisting of an
element, referred to as the \emph{head}\index{list!head} of the list,
and another list, referred to as the \emph{tail}\index{list!tail} of the list.
\end{quote}
%
The nil list is denoted by \ccode{[]},
which is read ``nil''\index{list!notation}.
A cons list, with head $h$ and tail $t$ is denoted by \ccode{$h$:$t$}.
The infix operator \ccode{:}, which is read ``cons'',
is right associative with precedence 5.
A list can also be denoted by enumerating its
elements\index{list!enumeration}, e.g.,
\ccode{[u,v,w]} is a list containing three elements,
\ccode{u}, \ccode{v} and \ccode{w}, i.e., it is just another
notation for \ccode{u:v:w:[]}.
The number of elements is arbitrary.
The elements are enclosed in brackets and separated by commas.

The following functions concatenate two lists
and reverse a list, respectively.  
The \ccode{Prelude} defines the first one as the infix operator \ccode{++}
and the second one, much more efficiently, as the operation \ccode{reverse}.
%
\begin{curry}
conc []     ys = ys
conc (x:xs) ys = x : conc xs ys

rev []     = []
rev (x:xs) = conc (rev xs) [x]
\end{curry}
%
Several \emph{ad hoc} notations available for lists are 
described in Sections~\ref{list-ranges} and~\ref{list-comprehensions}.

A key advantage of these special notations for lists is
a reduction of the number of parentheses needed to represent
list expressions in a program.
This claim can be easily verified by comparing the builtin
notation with the ordinary notation which was the subject of
Exercise~\ref{ex_userlist}.


\section{Strings}

Although \ccode{String} is a predefined type
(see Section~\ref{Predefined Types}), there are no special
operations on strings. The reason is that \ccode{String}\pindex{String}
is just another name for \ccode {[Char]}, i.e., strings
are considered as lists of characters. In addition, Curry
provides a handy notation for string constants, i.e.,
the string constant
%
\begin{curry}
"hello world"
\end{curry}
%
is identical to the character list
%
\begin{curry}
['h','e','l','l','o',' ','w','o','r','l','d']
\end{curry}
%
Thus, any operation applicable to arbitrary lists can also be applied
to strings. For instance, the prelude defines an infix operator
\ccode{++}\pindex{++} to concatenate lists and the function
\ccode{reverse}\pindex{reverse}
to reverse the order of all lists elements
(similarly to \code{conc} and \code{rev} in
Section~\ref{sec-lists}). Thus, we can also use them to operate on
strings:
%
\begin{prog}
Prelude> \userinput{"Hi"++"Hi"}
"HiHi"
Prelude> \userinput{reverse "hello"}
"olleh"
\end{prog}


\section{Tuple}

The word ``tuple''\index{tuple} is a generic name
for a family of related types.
A tuple in a program is similar to
a tuple in mathematics, i.e., a fixed
length sequence of values of possibly different types.
Examples of tuples are pairs and triples.
They could be defined by the programmer as follows:
%
\begin{curry}
data Pair a b = Pair a b

data Triple a b c = Triple a b c
\end{curry}
%
These types are polymorphic.  Observe the two occurrences of
the identifiers \ccode{Pair} and \ccode{Triple} in the above declarations.
The occurrence to the left names a type constructor,
whereas the occurrence to the right names a data constructor.
These symbols are \emph{overloaded}\index{overloading}.
However, this kind of overloading causes no problems since
type expressions are clearly separated from value expressions.
The type variables \ccode{a}, \ccode{b}$\ldots$ can be bound
to different types.

For example, the information system of a ``Big \& Tall'' shoe store
declares a function that defines the largest size and width of each model
\proghref{BigTall}{Program}:
%
\begin{curry}
data Width = C | D | E | EE | EEE | EEEE
largest "New Balance 495" = Pair 13 EEE
largest "Adidas Comfort"  = Pair 15 EE
...
\end{curry}
%
The language predefines tuples and denotes them with a special
notation similar to the standard mathematical notation.
Using predefined tuples, the above function is coded as:
%
\begin{curry}
largest "New Balance 495" = (13,EEE)
largest "Adidas Comfort"  = (15,EE)
...
\end{curry}
%
Tuples are denoted by a fixed-length sequence of comma-separated
values between parentheses.
There is no explicit data constructor identifier.
The type of a tuple is represented
as a tuple as well, e.g., the type of \ccode{largest}
can be defined as:
%
\begin{curry}
largest :: String -> (Int,Width)
\end{curry}
%

\section{Type classes}
\label{Type classes}


A type class defines a generic interface abstracting
some common feature over a variety of types.
For example, many numeric structures, such as integer,
vectors and matrices, allow the addition of elements in the structure.
We capture the property of ``being \emph{addable}'' as follows:
%
\begin{curry}
class Addable a where
  (+) :: a ->  a  -> a
\end{curry}
%
The variable \code{a} is a type parameter.
The identifier \ccode{+} is chosen because we are generalizing
addition, but it is not the usual addition of integers.
Any symbols would acceptable:

To say that some structure is \emph{addable} we use an \emph{instance}
declaration.  For example, the following declaration
states that the integers are addable
and the addition operation is the addition function defined in the Prelude.
%
\begin{curry}
instance Addable Int where
  x + y = x Prelude.+ y
\end{curry}
%
Next, we represent vectors (without an explicit data declaration)
as lists of \emph{addables} and we define the addition of vectors
as the component-wise addition of their elements:
%
\begin{curry}
instance Addable a => Addable [a] where
  xs + ys = zipWith (+) xs ys
\end{curry}
%
The symbol \ccode{+} refers to the same symbol defined in the
\code{Addable} type class and it is applied to the vectors' elements.
With these definition we can add two vectors as follows:
%
\begin{curry}
testv :: [Int]
testv = [1,2,3] + [4,5,6]
\end{curry}
%
\begin{exercise}
  % \label{instance}
  Define a matrix as follows:
  \\[2ex]
  {\tt  \phantom{WWW} data Matrix a = Matrix [[a]]}
  \\[2ex]
  and make it \emph{addable} using an instance declaration.
  \proghref{xinstance}{Program}
\end{exercise}

\section{Higher-Order Computations}
\label{Higher-Order}

The arguments of a function can be functions themselves.
This feature is banned or restricted by many programming languages.
E.g., in C only a \emph{pointer} to a function can be passed
as a parameter to another function. 
% For the same purpose, {C\hskip -0.1ex\raise 0.5ex\hbox{\tiny ++}}
% uses \emph{templates} and Java uses \emph{interfaces}.
% In Curry, no special construct or concept is necessary.

A function that takes an argument of function type is referred
to as a \emph{higher-order}\index{function!higher order} function.
Loosely speaking, a higher-order function is computation parameterized
by another computation.
We show the power of this feature with a simple example.
The function \ccode{sort}, shown below,
takes a list of numbers and sorts them in ascending order.
On non-empty arguments, the function \ccode{sort} recursively sorts the tail
and inserts the head at the right place in the sorted tail.
This algorithm becomes inefficient as lists grow longer,
but it is easy to understand
\proghref{InsertionSort}{Program}:\label{InsertSort}
%
\begin{curry}
sort []     = []
sort (x:xs) = insert x (sort xs)

insert x [] = [x]
insert x (y:ys) | x <= y    = x : y : ys
                | otherwise = y : insert x ys
\end{curry}
%
To sort a list in descending order or to sort a list
of a different type, a new function must be coded.

An alternative is to code a sort function where the ordering
criterion is an argument.
The overall structure of the function is the same.
The new argument, the first one of each function,
is denoted by \ccode{f}.
This argument is a function that takes two arguments
and returns \ccode{True} if and only if the first argument
must appear before the second argument in the output list
\proghref{HOInsertionSort}{Program}:\label{HOInsertionSort}
%
\begin{curry}
sort _ []     = []
sort f (x:xs) = insert f x (sort f xs)

insert _ x [] = [x]
insert f x (y:ys) | f x y     = x : y : ys
                  | otherwise = y : insert f x ys
\end{curry}
%
For example:
%
\begin{curry}
HOInsertionSort> sort (<=) [3,5,1,2,6,8,9,7]
[1,2,3,5,6,7,8,9]
HOInsertionSort> sort (>) [3,5,1,2,6,8,9,7]
[9,8,7,6,5,3,2,1]
\end{curry}
%
In the above expressions, 
the operators \ccode{<=} and \ccode{>} are the functional arguments.
The parentheses around them are necessary, since these
functions are identified by infix operators.
Without parentheses, the expression \ccode{sort <= [3,5,1,2,6,8,9,7]}
would test whether the left argument of \ccode{<=}
is smaller than the right argument, which is meaningless.

Observe that the first version of the \ccode{sort} function constrains
the elements of the input list to be numbers,
since these elements are arguments of \ccode{<=}.
In the second, higher-order version, the type 
of the elements of the input list is unconstrained.
Thus, the function can be applied to lists of any type
as long as a suitable ordering criterion
for the type of the list elements is provided.

Higher-order computations involve a functional argument.
Sometimes, the corresponding argument in a call, which is
a function, is referenced only in the call itself.
In this case, it is appropriate to use an anonymous
function\index{function!anonymous}.
For example, suppose that an elementary school information
system represents classes with a grade and a section.
The grade is a number in the range 1 through 5 and the
section is a letter, \code{a}, \code{b} ...
The following ordering criterion sorts the classes
in a  ``natural'' (lexicographic) order
\proghref{anon}{Program}:
%
\begin{curry}
sortClasses xs = sort lex xs
   where lex (x,y) (u,v) = x<u || x==u && ord y <= ord v 
\end{curry}
%
A more compact and informative formulation uses
an anonymous function as follows
\proghref{anon}{Program}:
%
\begin{curry}
sortClasses xs = sort (\(x,y) (u,v) -> x<u || x==u && ord y <= ord v) xs
\end{curry}
%
Observe that pattern matching is normally used in the definition
of the above anonymous function.

% \bigskip\noindent
% \todo{Exercise to sort lists of some type other than integers}

\section{Lazy Evaluation}
\label{sec-lazy-evaluation}

The \emph{evaluation}\index{evaluation}
of an expression $t$ is the process of
obtaining a value $v$ from $t$.
%
\begin{quote}
A \emph{value}\index{value!definition} is an expression consisting
only of builtin literals and/or data constructors and/or variables.
\end{quote}
%
The value $v$ is obtained from $t$ by replacing an instance of
the left-hand side of a rule with the corresponding
instance of the right-hand side.
For example, referring to the function \emph{square} defined
in Section \ref{Basic concepts}:
%
\begin{curry}
square x = x * x
\end{curry}
%
an instance of $\emph{square}\;x$
is replaced with the corresponding instance of $x * x$.
For example, $4+\emph{square}\;(2+3)$ is replaced by $4+(2+3)*(2+3)$.

The evaluation of an expression $t$ 
proceeds replacement after replacement until an expression $v$
in which no more replacements are possible is obtained.
If $v$ is not a value, the evaluation fails,
otherwise $v$ is the result of a computation of $t$.
For example, the following function \emph{head}
computes the first element of a (non-null) list:
%
\begin{curry}
head (x:_) = x
\end{curry}
%
An attempt to evaluate \ccode{head []} fails, since
no replacement is possible and the expression is not a value
since it contains a function.

Often, an expression may contain several distinct replaceable subexpressions,
e.g., from $(2+3)*(2+3)$ we can obtain both $5*(2+3)$ and $(2+3)*5$.
Even a single subexpression may allow several distinct replacements
when non-deterministic functions are involved.
The order in which different subexpressions of an expression are
replaced is not determined by a program.
The choice is made by an evaluation
\emph{strategy}\index{evaluation!strategy}\index{strategy}.
The semantics of the language guarantees that any
value obtainable from an expression is eventually obtained.
This property is referred to as the
\emph{completeness}\index{evaluation!completeness}
of the evaluation.
To ensure this completeness, expressions must be evaluated
lazily\index{evaluation!lazy}\index{laziness}.
A lazy strategy is a strategy that evaluates a subexpression
only if its evaluation is unavoidable to obtain a result.
The following example clarifies this delicate point.

The following function computes the list of
all the integers beginning with some initial value $n$
\proghref{lazy}{Program}:
%
\begin{curry}
from n = n : from (n+1)
\end{curry}
%
An attempt to evaluate \ccode{from 1} aborts with a
memory overflow since the ``result'' would be the infinite term:
%
\begin{curry}
[1,2,3,...
\end{curry}
%
However, the function \ccode{from} is perfectly legal.
The following function returns the $n$-th element of a list:
%
\begin{curry}
nth n (x:xs) = if n==1 then x else nth (n-1) xs
\end{curry}
%
The expression \ccode{nth 3 (from 1)} evaluates to \code{3}
despite the fact that \ccode{from 1} has no (finite) value:
%
\begin{prog}
lazy> \userinput{nth 3 (from 1)}
3
\end{prog}
%
The reason is that only the third element of \ccode{from 1} 
is needed for the result.
All the other elements, in particular the infinite sequence
of elements past the third one, do not need to be evaluated.

Infinite data structures\index{data structure!infinite}
are an asset in the conjunction with lazy evaluation.
Programs that use infinite structures are often simpler
than programs for the same problem that use finite structures.
E.g., a function that computes a (finite) prefix of \ccode{[1,2,3,...}
is more complicated than \ccode{from}.
Furthermore, the functions of the program are less interpedendent
and consequently more reusable.
E.g., the following function, initially applied to
\code{0} and \code{1}, computes the (infinite) sequence of
the Fibonacci numbers:
%
\begin{curry}
fibolist x0 x1 = x0 : fibolist x1 (x0+x1)
\end{curry}
%
The function \ccode{nth} can be reused to compute the $n$-th
Fibonacci number through the evaluation of the expression
\ccode{nth $n$ (fibolist 0 1)}, e.g.:
%
\begin{prog}
lazy> \userinput{nth 5 (fibolist 0 1)}
3
\end{prog}
%
The evaluation strategy of the \pakcs{} compiler/interpreter,
which is used for all our examples, is lazy, but incomplete.
The strategy evaluates non-deterministic choices sequentially instead
of concurrently.

All the occurrences of same variables are shared.
This design decision has implications both on the efficiency
and the result of a computation.
For example, consider again the following definition:
%
\begin{curry}
square x = x * x
\end{curry}
%
The evaluation of say \code{square\,$t$} goes through
\code{$t$\,*\,$t$}.  Without sharing, $t$ would be evaluated
twice, each evaluation independent of the other.
If $t$ has only one value, the double evaluation
would be a waste.
If $t$ has more then one value, this condition will be
discussed in Section~\ref{where-clause},
sharing produces the same value for both occurrences.

\section{Local Definitions}
\label{sec:localdefinitions}

The syntax of Curry implicitly associates a scope
to each identifier, whether a function, a type, a variable, etc.
Roughly speaking the scope of an identifier is where in a program
the identifier can be used.
For example, the scope of a variable occurring in the left-hand side
of a rule is the rule itself, which includes the right-hand side
and the condition, if any.
In the following code:
%
\begin{curry}
square x = x * x

cube   x = x * square x
\end{curry}
%
the variable identified by \ccode{x} in the definition of \ccode{square}
is completely separated from the variable 
identified by \ccode{x} as well in the definition of \ccode{cube}.
Although these variables share the same name, they
are completely independent of each other.

Curry is \emph{statically scoped}\index{static scoping},
which means that the scope of an
identifier is a static property of a program, i.e.,
the scope depends on the textual layout of a program rather than on
an execution of the program.
%
\begin{quote}
The \emph{scope}\index{scope} of an identifier is the region of text
of a program in which the identifier can be referenced.
\end{quote}
%
In most cases, the programmer has no control on the scope of
an identifier---and this is a good thing.
The scope rules are designed to make the job of the programmer
as easy and safe as possible.
The context in which an identifier occurs determines the
identifier's scope.
However, there are a couple of situations where the programmer
can limit, by mean of syntactical constructs provided by the language,
the scope of an identifier.
Limiting the scope of an identifier is convenient in some situations.
For example, it prevents potential name clashes
and/or it makes it clearer that a function is
introduced only to simplify the definition of another function.
A limited scope,
which is referred to as a \emph{local scope}\index{scope!local},
is the subject of this section.

Curry has two syntactic constructs 
for defining a local scope: the \ccode{where} clause
and the \ccode{let} clause.
They are explained next.

\subsection{{\fontfamily{pcr}\fontseries{b}\selectfont Where} Clauses}
\label{where-clause}

A \ccode{where}\index{where clause}\pindex{where}
clause creates a scope nested within a rewrite rule.
The following example defines an infix operator, \ccode{**}, for
integer exponentiation
\proghref{power}{Program}:
%
\begin{curry}
infixl 8 **

a ** b | b >= 0 = accum 1 a b
   where accum x y z | z == 0    = x
                     | otherwise = accum aux (y * y) (z `div` 2)
           where aux = if (z `mod` 2 == 1) then x * y else x
\end{curry}
%
For example, $\code{2\,**\,5}=2^5=32$.
There are several noteworthy points in the above code fragment.
The scope of the function \ccode{accum} is limited to
the rewrite rule of \ccode{**}.
This is convenient since the purpose of the former
is only to simplify the definition of the latter.
There would be no gain in making the function \ccode{accum}
accessible from other portions of a program.
The function \ccode{accum} is \emph{nested}\index{function!nested}
inside the function \ccode{**}, which is \emph{nesting} \ccode{accum}.  

The rewrite rule defining \ccode{accum} is conditional.
Pattern matching of the arguments and non-determinism
can occur as well in local scopes.
Finally, there is yet another local scope nested
within the rewrite rule of the function \ccode{accum}.
The identifier \ccode{aux} is defined in this scope
and can be referenced from either condition or right-hand side
of the rewrite rule of the function \ccode{accum}.

The right-hand side of the rewrite rule defining \ccode{aux}
references the variables \ccode{x}, \ccode{y} and \ccode{z}
that are arguments of \ccode{accum} rather than \ccode{aux} itself.
This is not surprising since the scope of these variables
is the rewrite rule of \ccode{accum} and \ccode{aux}
is defined within this rule.

The identifier \ccode{aux} takes no arguments.
Because it occurs in a local scope\index{scope!local},
\ccode{aux} is considered
a local \emph{variable}\index{variable!local}
instead of a \emph{nullary function}.
The language does not make this distiction for non-local
identifiers, i.e., identifiers defined at the top level.
The evaluation of local variables differs from that of local functions.
All the occurrences of a variable,
whether or not local, share the same value.
This policy may affect both the efficiency of a program execution
and the result of computations involving non-deterministic
functions.  The following example clarifies this subtle point
\proghref{localvar}{Program}:
%
\begin{curry}
coin = 0
coin = 1

g = (x,x) where x = coin

f = (coin,coin)
\end{curry}
%
The values of \ccode{g} are \code{(0,0)} and \code{(1,1)} only,
whereas the values of \ccode{f} also include 
\code{(0,1)} and \code{(1,0)}.
The reason of this difference
is that the two occurrences of \ccode{coin} in the rule of \ccode{f}
are evaluated independently, hence they may have different values,
whereas the two occurrences of \ccode{x}
in the rule of \ccode{g} are ``shared,'' hence they have the same value.

There is one final important aspect of local scoping.
A local scope can declare an identifier already declared
in a nesting scope---a condition referred to as
\emph{shadowing}\index{shadowing}\index{scope!shadowing}.
An example of showing is shown below:
%
\begin{curry}
f x = x where x = 0
\end{curry}
%
The variable \ccode{x} introduced in the \code{where} clause
\emph{shadows} the variable with the same name introduced in
the rewrite rule left-hand side.
The occurrence of \ccode{x} in the right-hand side is
bound to the former.
Hence, the value \ccode{f\,1} is \code{0}.
This situation may be a source of confusion for the beginner.
The \pakcs{} compiler/interpreter detects this situation and
warns the programmer as follows
\proghref{shadow}{Program}:
%
\begin{prog}
Prelude> \userinput{:l shadow}
...
shadow.curry, line 1.3: Warning:
    Unused declaration of variable `x'
shadow.curry, line 1.15: Warning:
    Shadowing symbol `x', bound at: shadow.curry, line 1.3
\end{prog}
%
The second warning reports that the identifier in line 1,
column 15, the variable \ccode{x} in the local scope,
shadows some identifier(s) with the same name.
The first warning reports that
the identifier in line 1, column 3,
the variable \ccode{x} argument of \ccode{f},
is not used.  This is a consequence of its shadowing and
gives an important clue that the occurrence of \ccode{x}
in the right-hand side of the rewrite rule of \ccode{f}
is bound to the local variable rather than the argument.


\subsection{{\fontfamily{pcr}\fontseries{b}\selectfont Let} Clauses}
\label{let-clause}

A \ccode{let}\index{let clause}\pindex{let}
clause creates a scope nested within an expression.
The concept is very similar to a \ccode{where} clause,
but the granularity of the scope is finer.
For example, the program for integer exponentiation
presented earlier can be coded using \ccode{let} clauses
as well
\proghref{PowerLet}{Program}:
%
\begin{curry}
infixl 8 **

a ** b | b >= 0 = 
  let accum x y z | z == 0    = x
                  | otherwise =
                       let aux = if (z `mod` 2 == 1) then x * y else x
                       in  accum aux (y * y) (z `div` 2)
  in  accum 1 a b
\end{curry}
%
Using a \ccode{let} declaration is more appropriate than a
\ccode{where} declaration for the definition of operation \ccode{aux}.
With a \ccode{let} declaration, the scope of the identifier \ccode{aux}
is the right-hand side of the second conditional rule of
the function \ccode{accum} instead of the whole rule.

\subsection{Layout}
\label{sec-layout}

\index{layout}
By contrast to most languages,
Curry programs do not use a printable character to separate
syntactic constructs, e.g., one rewrite rule from the next.
Similar to Haskell,
Curry programs use a combination of an end-of-line and
the indentation of the next line, if any.
A Curry construct, e.g., a \ccode{data} declaration or a rewrite rule,
terminates at the end of a line, unless the following line
is more indented.
For example, consider the following layout:
%
\begin{curry}
f = g
 h...
\end{curry}
%
Since \ccode{f} starts in column 1 and \ccode{h} starts in column 2,
the right-hand side of the rule defining \ccode{f} consists in
the application of \ccode{g} to \ccode{h} to \ccode{...}\ %
By contrast, with the following layout:
%
\begin{curry}
f = g
h...
\end{curry}
%
the right-hand side of the rule defining \ccode{f} consists of
\ccode{g} only.
Since \ccode{h} starts in the same column as  \ccode{f},
this line is intended as a new declaration.

The layout style described above goes under the name
``off-side rule''\index{off-side rule}.
The examples of Sections~\ref{where-clause} and~\ref{let-clause}
shows how the off-side rule applies to \ccode{where} and
\ccode{let} clauses.

\section{Variables}

Most of the programs discussed so far are functional.
They declare data and/or define functions.
An execution of the program is the functional-like
evaluation of an expression.
Curry is a \emph{functional logic}\index{functional logic languages}
programming language.
It adds two crucial features to the model outlined above:
non-determinism, which was discussed in Section~\ref{Non-determinism},
and \emph{logic variables}\index{logic variable},
which are discussed in this section.

\subsection{Logic Variables}
\label{Logic variables}

A logic variable differs from the variables introduced by the left-hand
side of a rewrite rule.
A variable introduced by the left-hand side of a rewrite rule,
also called \define{pattern variable},
stands for any expression (of an appropriate type).
For example, the following definition:
%
\begin{curry}
head (x:xs) = x
\end{curry}
%
is read as ``for all expressions $x$ and $xs$ the head of (the list)
\code{(x:xs)} is \code{x}.''
Since Curry is strongly typed, the type of \code{xs} must be list,
otherwise the program would be invalid,
but no other conditions are imposed on \code{xs}. 

A \emph{logic variable} either is a variable occurring in an expression
typed by the user at the interpreter prompt or
it is a variable in the condition and/or right-hand side of
a rewrite rule which does not occur in the left-hand side.
We show an example of both.
The operation \ccode{==},
called \emph{Boolean equality}\index{Boolean equality},
is predefined in Curry.
Hence, one can (attempt to) evaluate:
%
\begin{prog}
Prelude> \userinput{z==2+2  where z free}
\end{prog}
%
Every variable in a query, such as \ccode{z} in the above example,
is a logic variable that initially is not bound to any value.
We will discuss shortly why queries with variables
may be useful and how variables are handled.
\\[1ex]
The second kind of logic variable is shown in the following
example:
%
\begin{curry}
path a z = edge a b && path b z   where b free
\end{curry}
%
The intuition behind the names tells that in a graph there exists a path
from a node $a$ to a node $z$ if there exists an edge from
the node $a$ to some node $b$ and a path from the node $b$
to the node $z$.
In the definition, both \ccode{a} and \ccode{z} are ordinary (rule)
variables, whereas \ccode{b} is a logic variable.
Variables, such as \ccode{b}, which occur in the condition
and/or right-hand side of a rule, but not in the left-hand side,
are also called \emph{extra variables}\index{extra variable}.
Extra variables are explicitly declared \ccode{free}
in a \ccode{where} or \ccode{let} clause as shown in the example,
or are anonymous.

\subsection{Evaluation}

The evaluation of expressions containing logic variables
is a delicate issue and the single most important feature
of functional logic languages.
There are two approaches to deal with the evaluation of
expressions containing logic variables:
\emph{residuation}\index{residuation} and
\emph{narrowing}\index{narrowing}.

Let $e$ be an expression to evaluate and $v$ a variable
occurring in $e$.
Suppose that $e$ cannot be evaluated because the value of
$v$ is not known.
Residuation suspends the evaluation of $e$.
If it is possible, we will address this possibility shortly,
some other expression $f$ is evaluated in hopes that the evaluation
of $f$ will bind a value to $v$.  If and when this happens,
the evaluation of $e$ resumes.
If the expression $f$ does not exists, $e$ is said to
\emph{flounder}\index{floundering} and the evaluation of $e$
fails.
For example, this is what would happen for the query we showed
earlier:
%
\begin{prog}
Prelude> \userinput{z == 2+2  where z free}
*** Warning: there are suspended constraints (for details: ":set +suspend")
\end{prog}
%
By contrast to residuation, 
if $e$ cannot be evaluated because the value of $v$ is not known,
narrowing guesses a value for $v$.
The guessed value is uninformed except that only values that
make it possible to continue the computation are chosen.

The operation \ccode{=:=},
called \emph{constrained equality}\index{constrained equality},
is predefined in Curry.
This operation is similar to the Boolean equality
discussed earlier except that it returns only \code{True}
(if the equality can be satisfied) or it fails,
i.e., it does not return \code{False} if both sides are not evaluable
to identical values.
Since this operation succeeds only if both sides have identical values,
it also binds logic variables if this is necessary to make the sides
identical.
Thus:
%
\begin{prog}
Prelude> \userinput{z =:= 2+2  where z free}
\{z=4\} True
\end{prog}
%
Therefore, this operation can be used to express an
equational constraint that must be satisfied to compute some result.

\subsection{Flexible vs. Rigid Operations}

Operations that residuate are called
\emph{rigid}\pindex{rigid}\index{function!rigid},
whereas operations that narrow are called
\emph{flexible}\pindex{flexible}\index{function!flexible}.
All defined operations are flexible
whereas most primitive operations,
like arithmetic operations, are rigid since guessing is not
a reasonable option for them.
For example, the prelude defines a list concatenation
operation as follows:
%
\begin{curry}
infixr 5 ++
...
(++)          :: [a] -> [a] -> [a]
[]     ++ ys  = ys
(x:xs) ++ ys  = x : xs ++ ys
\end{curry}
%
Since the operation \ccode{++} is flexible, we can use it to search for
a list satisfying a particular property:
%
\begin{prog}
Prelude> \userinput{xs ++ [3,4] =:= [1,2,3,4]  where xs free}
\{xs=[1,2]\} True
\end{prog}
%
On the other hand, predefined arithmetic operations like the addition \ccode{+}
are rigid. Thus, a call to \ccode{+} with a logic variable as an argument
flounders:
%
\begin{prog}
Prelude> \userinput{x + 2 =:= 4  where x free}
*** Warning: there are suspended constraints (for details: ":set +suspend")
\end{prog}
%
For \emph{ground expressions}\index{ground expression}\index{expression!ground},
i.e., expressions without logic variables, the flex/rigid status of
a function makes no difference.
In the context of concurrent/distributed object-oriented programming,
rigid user-defined functions can be useful.
For this purpose, there is a primitive operation
\code{ensureNotFree}\pindex{ensureNotFree}
that evaluates its argument and suspends if it is a logic variable.


\subsection{Programming}

Often, programming with variables leads
to conceptually simple, terse and elegant code
at the cost of an acceptable loss of efficiency.
The logic variables of a program and/or a query
are not much different from the variables that are typically
used to solve algebra or geometry problems.
In both cases, some unknown entities of the problem
are related to each other by expressions involving functions.
Narrowing allows us to evaluate these expressions---and
in the process to find values for the variables.
The simplest application, and the most familiar for
those used to solve algebra and geometry problems with variables,
is when the expression to evaluate is an equation.

In later chapters, we will discuss some problems that
are conveniently solved if one uses variables in computations.
Here we want to present a simple, but non-trivial, motivating example.
The problem is to parse a string that represents an expression.
To keep the example small, our expressions are functional terms
whose syntax is defined by:
%
\begin{prog}
term       ::= identifier
            |  identifier '\code{(}' args '\code{)}'
args       ::= term
            |  term '\code{,}' args
identifier ::= \progit{any non-null string of alphabetic characters}
\end{prog}
%
For example, \code{"f(g(a,b))"} is a term described
by the above syntax.
When a term is represented as a string,
answering questions such as how many arguments \ccode{g} has,
is more complicated and less efficient than it needs to be.
A parser converts a term from its string representation into
a data structure that
makes easy and efficient answering questions of that kind.
Thus, the first step to build a parser is to design a suitable type
to represent a term. Our choice is:
\proghref{TermParser}{Program}:
%
\begin{curry}
data Term = Term String [Term]
\end{curry}
%
Note that the occurrence of \ccode{Term} to right of the \ccode{=}
character is a data constructor, whereas the two other occurrences
are type constructors.  The \ccode{Term} identifier
is \emph{overloaded}\index{overloading} by the declaration.

Using this data structure, we represent a function identifier with
a string and the function's arguments with a list of terms.  For
example, the term \code{"f(g(a,b))"} would be represented as
\ccode{Term "f" [Term "g" [Term "a" [],Term "b" []]]}.
The following operation parses a term
\proghref{TermParser}{Program}:
%
\begin{curry}
parseTerm (fun++"("++args++")") | all isAlpha fun = Term fun (parseArgs args)
parseTerm s | all isAlpha s = Term s []
\end{curry}
%
The elements of the program relevant to our discussion are
the variables \ccode{fun} and \ccode{args}.
The functional pattern in the first rule acts similarly to solving the
equation \ccode{s =:= fun++"("++args++")"} in which \ccode{s}
is the argument of \ccode{parseTerm}.
The first condition of the first rule of the operation \ccode{parseTerm}
instantiates these variables and ensures that \ccode{fun}
is an alphabetic identifier.
The operation \ccode{isAlpha}, defined in the library \ccode{Data.Char},
ensures that its argument, a character, is alphabetic.
The operation \ccode{all} is defined in the \ccode{Prelude}.
The combination \ccode{all isAlpha} ensures that all the characters
of a string are alphabetic.

If a term has arguments, these arguments are parsed by the operation
\ccode{parseArgs}.
The overall design of this operation is very similar to that
of \ccode{parseTerm}.
In this case, though, a string is decomposed according to
different criteria
\proghref{TermParser}{Program}:
%
\begin{curry}
parseArgs (term++","++terms) = parseTerm term : parseArgs terms
parseArgs s = [parseTerm s]
\end{curry}
%
One could code more efficient versions of this parser.
This version is very simple to understand and it is the
starting point for the design of more efficient versions that will be
discussed later in this book.


\section{Input/Output}
\label{sec-io}

As we have seen up to now, a Curry program is a set of datatype
and function declarations. Functions associate result values
to given input arguments. However, application programs must
also interact with the ``outside'' world, i.e., they
must read user input, files etc.
Traditional programming languages addresses this problem
by procedures with side effects, e.g., a procedure \code{read}
that returns a user input when it is evaluated.
Such procedures are problematic in the context of Curry.
Firstly, the evaluation time of a function is difficult
to control due to the lazy evaluation strategy
(see Section~\ref{sec-lazy-evaluation}). Secondly,
the meaning of functions with side effects is unclear.
For instance, if the function \code{readFirstNum} returns
the first number in a particular file, the evaluation
of the expression \ccode{2*readFirstNum} yields different values
at different points of time (if the contents of the file changes).

Curry solves this problem with the ``monadic I/O'' concept
\index{monadic I/O} much like that seen in the functional language
Haskell \cite{Wadler97}.
In the monadic approach to I/O, a program interacting with
the outside world is considered as a sequence of actions that
change the state of the outside world. Thus, an interactive program
computes actions which are applied to a given state of the world
(this application is finally done by the operating system
that executes a Curry program).
As a consequence, the outside world is not directly
accessible but can be only manipulated through actions that change the world.
Conceptually, the world is encapsulated in an abstract datatype
which provides actions to change the world.
The type of such actions is \ccode{IO t}\pindex{IO}
which is an abbreviation for
\begin{curry}
World -> (t,World)
\end{curry}
where \ccode{World} denotes the type of all states of the outside world.
If an action of type \ccode{IO t} is applied to a particular world,
it yields a value of type \code{t} and a new (changed) world.

For instance, \code{getChar}\pindex{getChar}
of type \ccode{IO Char} is an action
which reads a character from the standard input whenever it is executed,
i.e., applied to a world. Similarly, \code{putChar}\pindex{putChar}
of type \ccode{Char -> IO ()} is an action which takes a character
and returns an action which, when
applied to a world, puts this character to the standard output
(and returns nothing, i.e., the unit type).
The important point is that values of type \code{World} are
not accessible to the programmer --- she/he can only create
and compose actions on the world.

Actions can only be sequentially composed, i.e., one can built a new
action that consists of the sequential evaluation of two
other actions. The predefined function \pindex{>>}
\begin{curry}
(>>) :: IO a -> IO b -> IO b
\end{curry}
takes two actions as input and yields an action as the result.
The resulting action consists of performing the first action followed
by the second action, where the produced value of the first action is
ignored. For instance, the value of the expression
\ccode{putChar 'a' >> putChar 'b'} is an action which prints
\ccode{ab} whenever it is executed. Using this composition operator,
we can define a function \code{putStrLn}\pindex{putStrLn}
(which is actually predefined in the prelude)
that takes a string and produces an action to print this string:
\begin{curry}
putStrLn []     = putChar '\n'
putStrLn (c:cs) = putChar c >> putStrLn cs
\end{curry}
If two actions should be composed and the
value of the first action should be taken into account
before performing the second action, the actions can be also composed by
the predefined function\pindex{>>=}
\begin{curry}
(>>=) :: IO a -> (a -> IO b) -> IO b
\end{curry}
where the second argument is a function taking the value produced
by the first action as input and performs another action.
For instance, the action
\begin{curry}
getChar >>= putChar
\end{curry}
is of type \ccode{IO ()} and copies, when executed, a character
from standard input to standard output.
Actually, this composition operator is the only elementary one
since the operator \ccode{>>} can be defined in terms of \ccode{>>=}:
\begin{curry}
a1 >> a2  =  a1 >>= \_ -> a2
\end{curry}
There is also a primitive ``empty'' action\pindex{return}
\begin{curry}
return :: a -> IO a
\end{curry}
that only returns the argument without changing the world.
Thus, one can define an ``empty'' action\pindex{done}
which returns nothing (i.e., the unit type) as
\begin{curry}
done :: IO ()
done = return ()
\end{curry}
%
Using these primitives, we can define more complex interactive programs.
For instance, an I/O action that copies all characters from
the standard input to the standard output up to the first period
can be defined as follows
\proghref{IOExamples}{Program}:
\begin{curry}
echo = getChar >>= \c -> if c=='.' then return ()
                                   else putChar c >> echo
\end{curry}
%
Obviously, such a definition is not well readable.
Therefore, Curry provides a special syntax extension for
writing sequences of I/O actions, called the
\emph{do notation}.\index{do notation}\pindex{do}
The do notation follows the layout style
(see Section~\ref{sec-layout}), i.e., a sequence of actions
is vertically aligned so that
\begin{curry}
do putChar 'a'
   putChar 'b'
\end{curry}
is the same as \ccode{putChar 'a' >> putChar 'b'}, and
\begin{curry}
do c <- getChar
   putChar c
\end{curry}
is just another notation for \ccode{getChar >>= \char92c\,->\,putChar c}.
Thus, the do notation allows a more traditional style of writing
interactive programs. For instance, the function \code{echo}
defined above can be written in the do notation as follows:
\begin{curry}
echo = do c <- getChar
          if c=='.'
            then return ()
            else do putChar c
                    echo
\end{curry}
As a further example, we show the definition of the I/O action
\code{getLine} as defined in the prelude.
\code{getLine} as an action that reads a line from the standard input
and returns it:\pindex{getLine}
\begin{curry}
getLine :: IO String
getLine = do c <- getChar
             if c=='\n'
                then return []
                else do cs <- getLine
                        return (c:cs)
\end{curry}
Curry also provides predefined I/O actions for reading files
and accessing other parts of the environment.
For instance, \ccode{readFile f}\pindex{readFile}
is an action which returns the contents
of file \code{f} and \ccode{writeFile f s}\pindex{writeFile}
is an action writing string
\code{s} into the file \code{f}. This allows us to define a function
that copies a file with transforming all letters into uppercase ones
in a very concise way (\code{toUpper}\pindex{toUpper}
is defined in the standard character library \code{Data.Char} and converts
lowercase into uppercase letters)
\proghref{IOExamples}{Program}:
\begin{curry}
convertFile input output = do
  s <- readFile input
  writeFile output (map toUpper s)
\end{curry}
% To avoid introducing map in this chapter,
% consider next version or add function to Char library
% or explain map informally (as for ``all'' in parser
% example.
%
% convertFile input output =
%    do s <- readFile input
%       writeFile output (capitalize s)
%    where capitalize [] = []
%          capitalize (c:cs) = toUpper c : capitalize cs
% 
The function \code{toUpper}, defined in the library \ccode{Data.Char},
takes a character. If the character is lower case and alphabetic,
then it returns it in upper case, otherwise it returns it unchanged.
The operation \ccode{map} is defined in the \ccode{Prelude}
and discussed in detail in Section~\ref{ho-list-functions}.
The combination \ccode{map toUpper} transforms all the characters
of a string to upper case.

The monadic approach to input/output has the advantage
that there are no ``hidden'' side effects---any interaction
with the outside world can be recognized by the \code{IO} type of the
function. Thus, functions can be evaluated in any order
and the only way to combine I/O actions is a sequential one, as
one would expect also in other programming languages.
However, there is one subtle point. If a function computes
non-deterministically different I/O actions,
like in the expression \ccode{putStrLn (show coin)}
(see Section~\ref{where-clause} for the definition
of the non-deterministic function \code{coin};
\code{show}\pindex{show} is a predefined function that
converts any value into a string),
then it is not clear which of the alternative actions
should be applied to the world. Therefore, Curry requires
that \emph{non-determinism in I/O actions must not occur}.
For instance, we get a run-time error if we evaluate the
above expression:
\begin{prog}
localvar> \userinput{putStr (show coin)}
ERROR: non-determinism in I/O actions occurred!
\end{prog}
%
One way to ensure the absence of such errors is the
encapsulation of all search between I/O operations, e.g.,
by using set functions.

\begin{exercise}
Define an I/O action \code{fileLength} that reads the name
of a file from the user and prints the length of the file,
i.e., the number of characters contained in this file.
\proghref{exfilelength}{Answer}
\end{exercise}

% Describe in separate chapters:
% \begin{itemize}
% \item partial applications, currying, sections
% \item encapsulated search
% \item Module System
% \item Type System
% \end{itemize}


%%% Local Variables: 
%%% mode: latex
%%% TeX-master: "main_pdf"
%%% End: 
