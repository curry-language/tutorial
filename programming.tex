\chapter{Programming in Curry}

\section{Overview}

\emph{Lists} and \emph{trees}
are datatypes frequently used in programming.

\begin{itemize}
\item{} 
A \emph{list}\index{list} abstracts a sequence of elements.
The elements of a list are implicitly ordered by the list structure.
Therefore, a list is a convenient representation for queues, stacks
and other linear structures.
As list can also be used for representing collections, typically
unordered, such as a set, by ignoring or hiding the implicit
order of the elements.
\item{} 
A \emph{tree}\index{tree} is a multibranching
data structure that abstracts a hierarchy of values or conditions.
It naturally represents taxonomies or classifications and
therefore it is useful for problems involving search.
Trees come in many variants, but all consists of a value
called \emph{root}\index{root} and some subtrees called
\emph{children}\index{child}.
Similar to lists, trees can be used for representing collections.
\end{itemize}
%
This section describes in some detail both these datatypes and how
they help solve some typical problems, e.g., sorting a
collection of elements or searching for an element in a collection.

\section{Lists}

\subsection{Notation}

A \emph{List}\index{list!definition} is a simple algebraic polymorphic
datatype defined by two constructors conventionally referred to
as \emph{Nil}\index{Nil} and \emph{Cons}\index{Cons}.
Within the Curry language, the datatype ``\emph{List of a}''
would be declared as:
%
\begin{curry}
data List a = Nil | Cons a (List a)
\end{curry}
%
Because lists are one of the most frequently used types in
functional, logic and functional logic programs,
many languages offer several special notations\index{list!notation} for lists.
In Curry, the type ``\emph{List of a}'',
where \emph{a} is a type variable that stands for any type,
is predefined and denoted by \code{[a]}.
Likewise, \code{[]} denotes the constructor \emph{Nil}, the empty list,
and \ccode{:}\pindex{:} denotes the constructor \emph{Cons}, which
takes an element of type \code{a} and a list of \code{a}'s.
Thus, with a syntax that is \emph{not} legal in Curry, but
is quite expressive, the above declaration would look like:
%
\begin{curry}
data [a] = [] | a : [a]
\end{curry}
%
The expression \code{(u:v)} denotes the list with the first element \code{u}
followed by the list \code{v}.
The infix operator \ccode{:}, which read ``cons'',
is predefined, right associative and has precedence 5.
This implies that \code{u:v:w} is parsed as \code{u:(v:w)}.

A list can also be denoted by enumerating its
elements\index{list!enumeration}, e.g.,
\ccode{[u,v,w]} is a list containing three elements,
\ccode{u}, \ccode{v} and \ccode{w}, i.e., it is just another
notation for \ccode{u:v:w:[]}.
This notation can be used with any number of elements.
The elements are enclosed in brackets and separated by commas.
This notation has several advantages over the standard algebraic
notation: lists stand out in a program and references to lists
are textually shorter.  In particular, the number of parentheses
occurring in the text is reduced.
This claim can be easily verified by comparing the built-in
notation with the ordinary notation.
% which was the subject of Exercise~\ref{ex_userlist}.
%%% Should the exercise be moved here? 

The type list is polymorphic, which means that
different lists can have elements of different types.
However, all the elements of a particular list must have the same type.
The following annotated examples show this point
\proghref{lists}{Program}:
%
\begin{curry}
-- list of integers
digits = [0,1,2,3,4,5,6,7,8,9]

-- list of characters, equivalent to "Pakcs", print with putStr
string = ['P','a','k','c','s'] 

-- list of list of integers
matrix = [[1,0,2],[3,7,2],[2,8,1],[3,3,4]]
\end{curry}
%
Other special notations available for lists are described in 
Sections~\ref{list-ranges} and~\ref{list-comprehensions}.

\subsection{Inductive Definitions}

Many elementary functions on lists are defined by an induction similar
to that available for the naturals.
The cases of the induction are conveniently defined by
different rules using pattern matching.
For lists, the base case involves defining a function for \code{[]}
whereas the inductive case involves defining the function for
a list \code{(u:v)} under the assumption that the value of the
function for \code{v} is available.
In a program, this is expressed by a recursive call.
The function that counts the number of elements of a list
is emblematic in this respect:
%
\begin{curry}
len []    = 0
len (u:v) = 1 + len v
\end{curry}
%
For computing the length of a list, the value of \code{u} is irrelevant and
\code{u} should be replaced by an anonymous variable in the above
definition.
%
\begin{exercise}
Code an inductively defined function that takes a list of integers and
returns the sum of all the integers in the list.
Hint: the function should return \code{0} for an empty list.
\proghref{sumlist}{Answer}
\end{exercise}
%
The prelude defines many useful functions on lists,
e.g., \ccode{++} for concatenation, \ccode{!!} for indexing,
i.e., \code{(l!!i)} is the $i$-th (starting from 0) element of \code{l},
etc.  We will use some of these functions, after providing a brief
explanation, in this section.  We might also re-define some functions
already available in the prelude or other libraries
when they make good examples.  E.g., the function \code{len}
discussed above is equivalent to the function \code{length}
of the prelude.
In Section~\ref{basic-list-function}, we will present the
most important list functions available in the prelude.

Functions inductively defined are easy
to code, understand and evaluate.  Sometimes they may be inefficient.
Below are two definitions of a function to reverse a list.
For long lists, the second one is much more efficient.
%
\begin{curry}
slowRev [] = []
slowRev (u:v) = slowRev v ++ [u]

fastRev l = aux l []
  where aux [] r = r
        aux (u:v) r = aux v (u:r)
\end{curry}
%
A function inductively defined performs a ``traversal'' of its argument.
During this traversal some computation is performed on each
element of the list---this is referred to \emph{visiting}
a \emph{cons}---and the result combined with a
recursive invocation of the function.
Loosely speaking, the visit
can be performed either \emph{before} the recursive call,
or \emph{after}, or \emph{both}.
The following example shows how to subtract the minimum element of
a list of integers from all the elements of the list.
The function performs a single traversal of its argument.
The minimum of the list is computed (as much as feasible) before
the recursive call.
The subtraction is computed after the recursive
calls (otherwise the minimum could not be known)
\proghref{submin}{Program}:
%
\begin{curry}
submin []     = []
submin (x:xs) = fst (aux (x:xs) x)
  where aux []     m = ([],m)
        aux (y:ys) m = let (zs,n) = aux ys (min y m)
                        in (y-n:zs,n)
\end{curry}
%%% Would a diagram of the execution help?
%
The function \code{fst}, which returns the first element of a pair,
is defined in the prelude.
The function \code{min}, which returns the minimum of two
integers, is defined in the standard prelude.

More complicated computations may lead to more complicated
inductive definitions. A discussion on the structure
and the design of inductively defined function is in~\cite{Antoy92ALP}.
%
\begin{exercise}
Code an inductively defined function that transposes a matrix
represented by a list of lists (all of the same length).
\proghref{transpose}{Answer}
\end{exercise}
%%% Show examples of input and output 
%
There are a couple of noteworthy alternatives to directly defining
inductive functions.  One involves higher-order list functions.
Some of these functions are presented in Section~\ref{ho-list-functions}.
The other involves narrowing.
Lists are a particularly fertile ground for narrowing.
Below are two definitions of the function that computes the
last element of a list.  The first definition is inductive,
whereas the second is narrowing-based. 
%
\begin{curry}
inductLast [x]     = x
inductLast (x:y:z) = inductLast (y:z)

narrowLast (_++[e]) = e
\end{curry}
%
Observe that \code{narrowLast} computes the result by solving the
equation \ccode{l == \_++[e]} in which \ccode{l} is the argument of
the call.
%

\subsection{Ranges}
\label{list-ranges}

A special notation is available to define lists containing
\emph{ranges}\index{list!ranges} of integers.
The most common of this notation is \ccode{[$e_1$\,..\,$e_2$]}\pindex{\lbrack{}u..v\rbrack{}}
which denotes the list \ccode{[$e_1,e_1+1,e_1+2,\cdots,e_2$]}.
For example:
%
\begin{prog}
Prelude> \userinput{[2\,..\,5]}
Result: [2,3,4,5] ? 
Prelude> 
\end{prog}
%
Similarly, the expression \ccode{[$e$\,..]}\pindex{\lbrack{}u..\rbrack{}}
denotes the \emph{infinite}\index{infinite structures}
list of all the integers starting from $e$.
This list cannot be printed in its entirety,
but it can be used in a program
if only a finite portion of the list is needed,
because the evaluation strategy is lazy.

The elements in the lists defined by the above expressions
are consecutive, i.e., the distance between adjacent elements is one.
The above expressions can be generalized to produce lists
where the distance between adjacent elements is a constant greater than one.
This distance is inferred from the first two elements of the expression.
For example:
%
\begin{prog}
Prelude> \userinput{[2,\,6\,..\,20]}\pindex{\lbrack{}u,v..w\rbrack{}}
Result: [2,6,10,14,18] ? 
Prelude> 
\end{prog}
%
Likewise, \ccode{[2,\,6\,..]}\pindex{\lbrack{}u,v..\rbrack{}}
generates the infinite list \ccode{[2,6,10,14,$\ldots$]}.

Ranges can be defined using ordinary functions.
The prelude defines four functions whose names start with
\code{enumFrom}.  These functions define in the ordinary
syntax the notations for ranges.

\subsection{Comprehensions}
\label{list-comprehensions}

Another useful notation involving lists goes under the name of
\emph{list comprehension}\index{list!comprehension}.
A list comprehension is a notation to construct a list from
one or more other lists called
\emph{generators}.
It goes without saying that ranges are simple generators.
For example, the infinite sequence of square and triangular
numbers are obtained as follows
\proghref{comprehensions}{Program}:
%
\begin{curry}
squares   = [x * x | x <- [0 ..]]
triangles = [x * (x+1) `div` 2 | x <- [0 ..]]
\end{curry}
%
A \emph{generator}\index{comprehension!generator}
is an expression of the form
\emph{var}\code{\,<-\,}\emph{list}.
Generators can be nested and/or combined with \emph{guards}.
A \emph{guard}\index{comprehension!guard}
is a Boolean expression that filters the elements
produced by the generator.
For example, if \code{isPrime} is a predicate telling whether
an integer greater than 2 is a prime number, the following
comprehension is the sequence of the prime numbers
\proghref{comprehensions}{Program}:
%
\begin{curry}
primes = [x | x <- [2 ..], isPrime x] 
\end{curry}
%
In this example, the guard is the Boolean expression
\code{(isPrime x)}. The elements produced by the generator
are passed to the comprehension if and only if the guard holds.

Generators are considered to be nested from left to right.
The following example shows how to compute pairs where
the second component is not greater than the first
\proghref{comprehensions}{Program}:
%
\begin{curry}
lexPairs = [(x,y) | x <- [0 .. 3], y <- [x .. 3]]
\end{curry}
%
This simple example shows that the second generator
\code{(y<-[x\,..\,3])} is nested within
the first one, since it references the generated elements.
%
\begin{exercise}
Compute the Fibonacci sequence using a list comprehension.
Hint: compute a list of \emph{pairs} of numbers where
each pair contains two \emph{consecutive} Fibonacci numbers.
\proghref{fibocompr}{Answer}
\end{exercise}

\subsection{Basic Functions}
\label{basic-list-function}

The \pakcs{} compiler/interpreter of Curry is distributed
with the prelude, a collection of primitive and
fundamental types and functions, and with several libraries.
The prelude and some of these libraries contain
useful list functions.
In this section, we informally discuss some of these functions.
The \code{currydoc}\pindex{currydoc}\index{documentation}
documentation utility, which is
distributed with \pakcs, should be used for an exhaustive
up-to-date consultation of the content of these libraries.

\vspace*{2ex}

\begin{tabular}{@{}lll@{}}
{\bf Name} & {\bf Description} & {\bf Example(s)} \\[1.5ex]
\code{head} & First element of a list & \code{head\,[1,2]}$\;=\;$\code{1};
   \code{head\,[]} fails \\
\code{tail} & All the elements but the first &
   \code{tail\,[1,2]}$\;=\;$\code{[2]}; \code{tail\,[]} fails \\
\code{length} & Length & \code{length\,[1,2]}$\;=\;$\code{2} \\
\code{null} & Tell whether it is nil &
   \code{null\,[1,2]}$\;=\;$\code{False} \\
\code{++} & Concatenate two lists & \code{[1,2]++[3]}$\;=\;$\code{[1,2,3]} \\
\code{!!} & $n$-th element of a list & \code{[1,2]!!1}$\;=\;$\code{[2]};
   \code{[1,2]!!4} fails \\
\code{reverse} & Reverse the order of the elements &
    \code{reverse\,[1,2]}$\;=\;$\code{[2,1]} \\
\code{concat} & Concatenate all the lists of a list &
   \code{concat\,[[1,2],[3]]}$\;=\;$\code{[1,2,3]} \\
\code{take} & List of the first $n$ elements &
   \code{take\,\,2\,[1,2,3]}$\;=\;$\code{[1,2]} \\
\code{drop} & All elements but the first $n$ &
   \code{drop\,\,2\,[1,2,3]}$\;=\;$\code{[3]} \\
\code{and} & Boolean conjunction &
   \code{and\,[True,False,True]}$\;=\;$\code{False} \\
\code{or} & Boolean disjunction &
   \code{or\,[True,False,True]}$\;=\;$\code{True} \\
\code{elem} & Whether a value is in a list &
   \code{elem\,\,2\,[1,3,5]}$\;=\;$\code{False} \\
\code{nub} & Remove duplicates & \code{nub\,[1,2,2]}$\;=\;$\code{[1,2]} \\
\code{delete} & Remove the first occurrence of a value&
   \code{delete\,\,2\,[2,1,2]}$\;=\;$\code{[1,2]}; \\
   && \code{delete\,\,2\,[1]}$\;=\;$\code{[1]} \\
\end{tabular}

\vspace*{2ex}

\noindent
Many more functions that operate on lists are defined
in the libraries of the \pakcs{} distribution
(e.g., see the library \code{Data.List} which contains
the definition of \code{nub} and \code{delete} discussed above).
The above table is intended to give only a feeling
of what is available.

\subsection{Higher-order Functions}
\label{ho-list-functions}
\index{list!higher-order functions}
\index{higher-order!on lists}

Lists are commonly used to represent collections of elements.
Some computations of a list can be expressed by repeatedly
applying another, somewhat simpler, computation
to all the elements of the collection.
This section discusses some frequently occurring situations
of this kind.

The simplest case is when a list, which we refer to as the
\emph{result list}, is obtained from another list,
which we refer to as the \emph{argument list}, by applying
the same function, say \code{f}, to all the elements of the argument list.
This is easily accomplished by defining a new function, say
\code{flist} since its analogy to \code{f}, as follows:
%
\begin{curry}
flist []     = []
flist (x:xs) = f x : flist xs
\end{curry}
%
Although trivial, the definition of \code{flist} can be
avoided altogether using the function \code{map},
provided by the prelude.
The function \code{map} is higher-order in that it takes
as an argument the function, in this example \code{f},
that is applied to all the arguments of the list.
Thus, the function \code{flist} defined above
is the same as \code{map\,\,f}.
\\[1ex]
The following code, taken from the prelude, shows
the type and the definition of \code{map}\pindex{map}:
%
\begin{curry}
map          :: (a->b) -> [a] -> [b]
map _ []     = []
map f (x:xs) = f x : map f xs
\end{curry}
%
It can be seen that the first argument of \code{map}
is a function from any type \emph{a} to any type \emph{b}.
The second argument of \code{map} is a list whose elements
must have, of course, type \emph{a}.
The result is a list of type \emph{b}.
For example, suppose that \code{isEven} is a function
telling whether an integer is even.
Then, the expression \code{(map\,isEven\,[0,1,2,3])}
evaluates to \code{[True,False,True,False]}.

A second frequently used higher-order function on lists is \code{filter}.
As the name suggests, \code{filter} is used to filter
the elements of a list that satisfy some criterion
expressed by a predicate.
\\[1ex]
The following code, taken from the prelude, shows
the type and the definition of \code{filter}\pindex{filter}:
%
\begin{curry}
filter          :: (a -> Bool) -> [a] -> [a]
filter _ []     = []
filter p (x:xs) = if p x then x : filter p xs else filter p xs
\end{curry}
%
It can be seen that the first argument of \code{filter}
is a function from any type \emph{a} to \code{Bool},
i.e., a predicate.
The second argument of \code{map} is a list whose elements
must have, of course, type \emph{a}.
The result is again a list of type \emph{a}.
The elements of the result are the elements of the second
argument that satisfy the predicate.
For example, as before, suppose that \code{isEven} is a function
telling whether an integer is even.
Then, the expression \code{(filter\,\,isEven\,\,[0,1,2,3])}
evaluates to \code{[0,2]}.

The last higher-order function operating on lists that
we describe in this section is used to ``combine together''
all the elements of a list.
For example, a function that adds all the elements of
a list of integers can be defined using a higher-order function and
the ordinary addition on integers.
Several options should be considered, e.g., whether the
elements of a list are composed starting with the first or
the last one, whether the list can be empty and thus
a default value must be supplied, etc.
The prelude contains a family of functions, referred to
as \emph{folds} for this purpose. The names of these
functions starts with ``\code{fold}''\index{folding functions}.
\\[1ex]
The following code, taken from the prelude, shows
the type and the definition of \code{foldr}\pindex{foldr}:
%
\begin{curry}
foldr            :: (a->b->b) -> b -> [a] -> b
foldr _ z []     = z
foldr f z (x:xs) = f x (foldr f z xs)
\end{curry}
%
For example, functions that compute the sum, product and maximum
of all the elements of a list of integers are easily defined
through \code{foldr} as follows
\proghref{fold}{Program}:
%
\begin{curry}
sumList  = foldr (+) 0
prodList = foldr (*) 1
maxList  = \l -> foldr max (head l) (tail l)
\end{curry}
%
The last function is more complicated than the previous two,
because it is meaningful only for non-empty lists.
The function \code{foldr1}, defined in the prelude,
would simplify our definition of \code{maxList}.

\newcommand{\funset}[1]{#1\ensuremath{_{\cal S}}}
\subsection{Set Functions}
\label{Set Functions}

A non-deterministic function $f$ may return different results for
some combination of arguments.  The programmer has no control over
which of these results an invocation of $f$ returns.
Any result returned by $f$ is completely unrelated to any other result.
For some problems, it may be convenient to consider \emph{all}
the results returned by $f$ as a whole.
The device that returns \emph{all} the results returned by $f$
for some combination of arguments is called the \emph{set function} of $f$
and is typically denoted $\funset f$.\index{set function}
For example, given:
%
\begin{curry}
idOrInc :: Int -> Int
idOrInc x = x ? x+1
\end{curry}
%
the value of \code{\funset{idOrInc} x} is the set \code{\{x,x+1\}}
where the details of the representation of this set are irrelevant to our
discussion.

The definition of the set function of a function $f$
separates the non-determinism of $f$ from any non-determinism
of the arguments to which $\funset{f}$ is applied \cite{AntoyHanus09PPDP}.
In particular, $\funset{f}$ is deterministic for any $f$.
Referring to the example above, \code{\funset{idOrInc} (2\,?\,4)} evaluates
to \code{\{2,3\}} and \code{\{4,5\}} in which the non-determinism
originates entirely from the argument.

To ease the definition of set functions,
there is a library\index{set function!library}
\code{Control.Search.SetFunctions}.
This library defines both a suitable abstraction of a type \emph{set}
to represent the results of a set function application,
denoted \code{Values$\;\tau$} (where $\tau$ is the type of elements
in the set), and
a family of functions \code{set0}, \code{set1}, $\ldots$~
Gives a function $f$ of arity $n$,
the set function of $f$ is provided by the expression \code{set$n$ $f$}.
The first argument of \code{set$n$} must be the \emph{identifier}
of a function of arity $n$.
For instance, we can write the set function \code{\funset{idOrInc}}
with this library as \code{set1$\;$idOrInc}.
This set function has the type
%
\begin{curry}
set1 idOrInc :: Int -> Values Int
\end{curry}
%
%%% This is a problematic example.
%%% There are two potential sources of confusion:
%%% 1. A set function that takes a set as argument
%%% 2. Representing some set abstractly and some set with a list
As another example, consider the problem of computing all the subsets
of a set $S$.  This is called the powerset\index{powerset} of $S$. 
Let us represent a set with a list.
This representation requires some care to ensure that
duplicate elements in a list and
the order of the elements in a list are not observable.
We ignore these issues since they are irrelevant to our discussion.
The following non-deterministic function returns a
subset of its argument, a set represented as a list
\proghref{AllSubsets}{Program}:
%
\begin{curry}
subset :: [a] -> [a]
subset []     = []
subset (x:xs) = x:subset xs
subset (_:xs) =   subset xs
\end{curry}
%
In \pakcs{}, with the help of the library \code{Control.Search.SetFunctions},
we define the power set of a set as follows:
%
\begin{curry}
powerset :: [a] -> Values [a]
powerset s = set1 subset s  -- powerset is the set function of subset
\end{curry}
%
For example, the prerequisites for the undergraduate 
Computer Science courses at Portland State
are abstracted by 16 rules as follows
\proghref{CSIS}{Program}:
%
\begin{curry}
isPrereqOf 162 = 161
isPrereqOf 163 = 162
isPrereqOf 200 = 162
...
isPrereqOf 303 = 252
isPrereqOf 303 = 300
isPrereqOf 350 = 252
\end{curry}
%
The meaning is that, e.g., \code{162} is a direct prerequisite of both
\code{163} and \code{200} and that, e.g., 
both \code{252} and \code{300} are direct prerequisites of \code{303}.
Observe that \code{IsPrereqOf} is a many-to-many relation.

The function to compute \emph{all} the direct prerequisites
of a course is shown below
\proghref{CSIS}{Program}:
%
\begin{prog}
\funset{isPrereqOf} course = set1 IsPrereqOf course
\end{prog}
%


\subsection{Narrowing}

Narrowing is a convenient programming feature when dealing with lists.
Lists are frequently used to represent collections of elements.
Sometimes the problem is to find in a list either elements
or sublists that satisfy certain relationships.
The programmer can either code functions to compute these elements
or express the relationships using variables for these elements
and let narrowing compute the elements by instantiating the variables.
Generally, the latter leads to simpler and more declarative programs.

For example, consider a program that plays the game of poker.
A hand is represented by a list of 5 cards.
Suppose that the problem is to find whether 4 out of the 5 cards
have the same rank, i.e., the hand is a four-of-a-kind.
A narrowing-based solution removes one card from the hand
so that the remaining 4 cards have the same rank.
The following function takes a hand and returns a Maybe type.
If the hand is a four-of-a-kind, the function returns \emph{just}
the rank of the four cards, otherwise it returns \emph{nothing}.
\proghref{poker}{Program}:
%
\begin{curry}
isFour (x++[_]++z) | map rank (x++z) == [r,r,r,r] = Just r where r free
isFour'default _ = Nothing
\end{curry}
%
The card removed from the hand is represented by the anonymous
variable in the functional pattern.
This card is non-deterministically selected.
The remaining cards are represented by \code{x} and \code{z}.
Additionally, the condition of the first rule imposes that
all the cards in \code{x} and \code{z}
have the same rank, \code{r}.
The rank, too, is non-deterministically selected by solving
the equation \ccode{map rank (x++z) == [r,r,r,r]}.
If the condition succeeds, there is obviously a unique
value for all these variables.
If the first rule cannot be applied, it must be that
there are no four cards of the same rank in the hand.
The default rule reports this condition

The advantage of the narrowing-based approach
over more conventional approaches is that no instructions need to
be coded both to isolate the card that does not contribute to
the four-of-a-kind nor to find the rank of the four-of-a-kind
when it exists.

The above example is typical of situations in which a collection
contains elements that must satisfy a certain conditions.
Since lists are implicitly ordered, conditions involving
the \emph{position} of elements in a collection can also be
conveniently expressed using narrowing.
% We will see an example of this kind in a program to
% solve the $n$-queens puzzle.

\begin{exercise}
Similar to the example just discussed, code a function
that tells whether a hand in a game of poker is a
\emph{full house}.
Hint: \proghref{Cards}{\code{Cards.curry}}
defines suits, ranks, etc.
\proghref{fullHouse}{Answer}.
\end{exercise}

\section{Trees}

We discuss two common variants of trees and show some typical functions
of each variant.

\subsection{Binary Search Trees}

A \emph{binary tree}\index{binary tree} is either of the
following: a singleton value called \emph{empty}\index{binary tree!empty}
(also \emph{null} or \emph{leaf}) or a pair, called a
\emph{branch}\index{binary tree!branch}, consisting of two
binary trees called the \emph{left} and the \emph{right child}.
Most often, a branch is a triple rather than a pair
which in addition to the children
also stores a value of some generic set.  This value
is called the \emph{root}\index{root} or \emph{decoration}
of the tree.

\medskip\noindent
A (decorated) binary tree is declared in Curry as:
%
\begin{curry}
data BinTree a = Leaf | Branch a (BinTree a) (BinTree a)
\end{curry}
%
where \code{BinTree} is a \emph{type constructor}
\index{data declaration!type constructor}\index{type constructor}
and $a$ is its parameter and stands for the type of the
decorations.
A popular variant of binary trees is called a
\emph{binary search tree}\index{binary search tree}.
The set of decorations is totally ordered and the decoration
of a non-empty binary search tree is greater than all the decorations
in the left child and smaller than all the decorations
in the right child.
This condition prevents repeated decorations in a binary search tree.
The following function inserts a value in a binary search tree
in such a way that the result is again a binary search tree
\proghref{binsearchtree}{Program}:
%
\begin{curry}
insert x Leaf = Branch x Leaf Leaf
insert x (Branch d l r)
  | x < d = Branch d (insert x l) r
  | x > d = Branch d l (insert x  r)
  | otherwise = Branch d l r
\end{curry}
%
An in-order traversal\index{tree!traversal} of any binary search tree
produces the sorted list of the decorations of the tree.
%
\begin{curry}
inorder Leaf = []
inorder (Branch d l r) = inorder l ++ [d] ++ inorder r
\end{curry}
%
\begin{exercise}
Sort a list without repeated values by constructing
a binary search tree from the list and then traversing the
tree
\proghref{treesort}{Answer}.
\end{exercise}

\subsection{Trie Trees}

An \emph{ordered tree}\index{ordered tree} is a pair
in which the first component is an element of a set,
called the set of decorations, and the second component
is a sequence of ordered trees.

\medskip\noindent
An ordered tree is declared in Curry as:
%
\begin{curry}
data Tree a = Tree a [Tree a]
\end{curry}
%
where $a$ is a parameter that stands for the type of the
decorations and the identifier ``\emph{Tree}'' is overloaded to
denote both a type constructor (1st and 3rd occurrences)
and a data constructor (2nd occurrence).

A \emph{trie} is a tree that stores a mapping from keys,
typically strings, to values and is characterized by sharing
key prefixes.  We present a simple variant where only keys
are stored.  This variant is useful to efficiently
represent a dictionary and tell whether a string is
in the dictionary.
%
%TODO re-add the trie for the htmlversion using something else
%TODO or wait for xymatrix support
\begin{pdfversion}
For example, a trie storing the strings ``car'', ``care''
and ``carton'' stores the letter `c', `a', and `r', only
once as shown below.  The distinguished symbol `$\bullet$'
terminates a string.

%\hspace*{2in}
$$
\xymatrix@1@C=6pt@R=6pt@M=1pt{
  & c \ar@{-}[d] \\  
  & a \ar@{-}[d] \\  
  & r \ar@{-}[dl]\ar@{-}[d]\ar@{-}[dr] \\  
  \bullet & e \ar@{-}[d] & t \ar@{-}[d] \\
  & \bullet & o \ar@{-}[d] \\  
  & & n \ar@{-}[d] \\  
  & & \bullet
}
$$
\end{pdfversion}

\noindent
We declare the type of a trie as follows:
%
\begin{curry}
type Trie = [Tree Char]
\end{curry}
%
The following function inserts a string in a trie
sharing whatever prefix of the string might already be present in the trie.
The distinguished symbol that terminates a string is a period.
\proghref{TrieInsert}{Program}
%
\begin{curry}
insert :: String -> Trie -> Trie
insert []     t  = (Tree '.' [] : t)
insert (w:ws) [] = [Tree w (insert ws [])]
insert (w:ws) (Tree c cs : ts)
  | ord w < ord c = insert (w:ws) [] ++ (Tree c cs : ts)
  | ord w > ord c = Tree c cs : insert (w:ws) ts
  | otherwise     = Tree c (insert ws cs) : ts
\end{curry}
%
\begin{exercise}
Code functions to build a trie from a list of words
and to print all the words in the trie
\proghref{TrieBuild}{Answer}.
\end{exercise}
%
\begin{exercise}
Code a function that takes a word and a trie and tells
whether or not the word is in the trie
\proghref{TrieSearch}{Answer}.
\end{exercise}

% \begin{itemize}
% \item Lists, Trees,
% \item Sorting, Searching, Solving, Text Processing, Parsing
% \item Patterns
% \end{itemize}
% 
% Inductive definitions \\
% Recursion: defining functions by pattern matching \\
% Map, fold \\
% Data structures: priority queues, circular queues \\
% Problem solving: n-queens, missionaries and cannibals, etc. \\

%%% Local Variables: 
%%% mode: latex
%%% TeX-master: "main_pdf"
%%% End: 
